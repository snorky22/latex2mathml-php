\documentclass[a4paper,10pt]{article}
\usepackage{amsmath,amssymb,amsthm,epsfig,graphics,hyperref}
\usepackage{caption}
\usepackage{subcaption}

\newtheorem{definition}{Definition}
\newtheorem{corollary}{Corollary}
\newtheorem{proposition}{Proposition}

\newtheorem{problem}{Problem}
\newtheorem{solution}{Solution method}

\newtheorem{theorem}{Theorem}
\newtheorem{lemm}{Lemma}
\newtheorem{coro}{Corollary}

\newtheorem{rema}{Remark}
\newtheorem{assumption}{Assumption}

\begin{document}
\title{Empirical averaging in deterministic optimal control}
\author{François Chaplais}
\maketitle

\date{\today}

\begin{abstract}
 A windowed averaged scheme is defined for general control
 systems. The same method is used to average costs in optimal
 control problems (OCPs). A numerical parameter $\alpha$ can be computed, which
 expresses the distance between the original system and the
 averaged system in a weak sense.

 Then, if we use the optimal control of the averaged OCP in the
 original OCP, the suboptimality of the control is bounded by an
 expression of the form $C\alpha^{2}$.
\end{abstract}

\section{Introduction}
We consider here optimal control problems which feature fast
dependency on time. This is the case, for instance, if the problems
depend on data which features high frequency phenomenons, i.e.,
features that happen at a much smaller time scale than the response
time of the dynamics.

This time scaling has consequences on the numerical solving of these
control problems. Typically, the discretization step used in a
numerical method will be determined by the fastest phenomenon that is
part of the system. In our settings, this is the sampling rate of the
input data, which is much higher than the time constant of the
dynamical system. We present in this paper a flexible and efficient
way to approximately solve the original control problem using an
averaging scheme. In this scheme, the state is sampled at a rate which is
determined by the time constant of the system, and not the fast
evolving features of the input data.

When the fast data is periodic with respect to time, a classic
approach to the solving of differential equations is averaging.
Historically, the method of averaging was introduced to study the
motion of celestial bodies by solving a simple two body equation
which is perturbed by the influence of other bodies (see the
section in \cite{sanders} about the history of averaging). As
developed in \cite{poincare}, the framework was that of the
perturbation of an orbit by the small influence of a periodic input.
Averaging was then generalized in a geometric framework, notably in
\cite{Arnold-celestial,ArnoldODE}. A comprehensive book on the subject
is \cite{sanders}. In this framework, the faster oscillating phenomenon is
either periodic or satisfies some kind of ergodicity assumption.

The previous references consider a non controlled dynamical systems.
We turn our attention now to controlled systems, and, in particular,
optimal control problems (OCPs).

In this perspective, we add to the differential equation an extra parameter $u$ which is the open
loop control of the system. If we have non-linearities in the
dynamics, common sense dictates that averaging be performed {\em
after} the control is introduced, by contrast to the approach where
the control would be introduced {\em after} the differential system
is averaged. Indeed, the high frequency content of the control may
interact with the high frequency content of the data; therefore the
averaging must be applied once this interaction is taken into account.

In the case of ordinary differential equations, the approximation value of the
averaging method is determined by the proximity of the nominal and
averaged trajectories.

However, in the case of controlled systems, there are as many trajectories to
consider as there are different controls. Fortunately, in optimal
control, there is a simple way to evaluate the performance of an approximation method:
\begin{itemize}
	\item We first consider the optimal cost $J_{0}$ for the nominal
	control system.

	\item On the other hand, we considered the optimal control
	$u^{*}$ for the approximated control system, and we use it to
	control the {\em nominal} system. This generates a cost $J^{*}$.

	\item Naturally, $J^{*}$ is greater than $J_{0}$. As a
	consequence, we can evaluate the
	quality of the approximation method by considering the loss of optimality
	$J^{*}-J_{0}$ which is non negative.
\end{itemize}

Let us review some previous work on this topic.

An early work \cite{balachandra} applies averaging to two point
boundary value problems, but its application to optimal control is
limited to what is essentially the linear quadratic case.
In \cite{chaplais}, the method of averaging is applied to optimal
control, both in open loop (the system is then periodic with respect
to the fast time), and in closed loop
(which is a study of the Hamilton-Jacobi equation under an ergodicity
assumption). The study of the HJB equation is improved in
\cite{Barron1993}. Observe that, in these two references, the horizon
is finite and the oscillatory input is fast.

References
\cite{GaitsgoryCDC2006,GaitsgorySIAM2006,GaitsgorySIAM2008,GaitsgoryCDC2009}
consider an optimal control problem ``in the long run'', that is,
with slow and ``normal'' time scales, using
averaging techniques. The convergence of the optimal cost is proved,
but there is no estimation of the loss of optimality that the
approximation method produces.

Reference \cite{Ogulenko} constructs the method of averaging for controlled systems on time scales,
and determines the algorithm of correspondence of controls over the
original and averaged systems. It is an interesting work, but we
shall prove in this article that the order of efficiency of the method is
under-estimated (it should be two).

As we pointed out before, the application of averaging to celestial
mechanics is a classic. It has been naturally transposed to the
context of optimal control. Specifically, the optimal control of low thrust engines
in space have been studied from a geometric view point and in the
periodic case. Such a approach has been adopted in
\cite{bonnard-BSM,Bonnard-Caillau2006,BonnardCaillauIHP}. Averaging
has also been used for similar problems
\cite{EpenoyGeoffroy1997,EpenoyEclipse} in a spirit that is close to
\cite{chaplais}. All of these references assume that the fast
dynamics are either periodic or satisfy an ergodicity assumption.

There are applications of averaging in optimal control that are more
data driven (by contrast to autonomous celestial mechanics). Energy
management of thermal systems consider the optimal control of the
temperature of buildings \cite{robillart}. Naturally, this heavily depends of the
weather data, which naturally features fast oscillating components
(i.e. fast with respect to the building's inertia).

Reference \cite{lepreux} considers a buoy which is subject to the
oscillating influence of waves.

Averaging has been also used is stochastic optimal
control, notably of Markov chains (see for instance
\cite{hernandez-lasserre,VargasCDC2009,TsaiHaddaACC1994}). Indeed,
when there exists a cycle in a discrete state Markov chain that has
high transition probabilities, then this cycle is gone through very
fast and averaging can be applied. An alternative to the classical
averaging scheme is ergodic theory \cite{Buckdahn}.

In this article, we consider optimal control problems which are
influenced by some fast perturbation data (with respect to the time
constant of the system). The approach is somehow different from what
exists in the litterature. In particular, there is no state space
representation of the process that generates the fast phenonmena
featured in the system. No periodicity is assumed, either.

Here we consider, as far as rapidly oscillating inputs are
considered, that the behavior of the system is similar to the response of
a single integrator. This is why we first define our averaging
method by the effect of integration on general signals. The averaged
signal
is defined by its successive averages on consecutive intervals. This
transformation can be from various points of view:
\begin{itemize}
	\item In the frequency domain, this averaging method is low
	pass filter, although not a very good one
	\cite{LiTan2008,Stranneby2004,Oppenheim2010}. Indeed, it is a Finite
	Impulse Response filter, and thus its transfer function is not
	compactly supported.

	\item In the time domain, this averaging operator is an
	orthogonal projection on the Haar basis. More efficient
	approximation methods exist in the wavelet framework \cite{Mallat98}; however
	specific tools must be used for compactly supported signals
	\cite{CohenDV:94,chaplais-denoising} and, in practice, their have
	boundary effects that are not desirable when noise is present.

	\item From a statistical point of view, this method can viewed
	as the computation of a sequence of empirical expectations,
	typically for a non stationnary signal. This why we call our
	method {\em empirical} averaging.
\end{itemize}

From this approximation technique we derive a similar one for
ordinary differential equations, control systems, and ultimately
optimal control problems.

In all of these contexts, the performance of the averaging method is
evaluated by the difference between the output of the reference
signal and the averaged one, through an integrator. The maximum value
for this difference is called $\alpha$.

It is a very simple thing to evaluate and as a consequence, it may be
applied to a wide range of situations.

An important feature of this problem is that the mesh size of the two
point boundary value problem is not dictated by the fast data, but
rather by the time constant of the system (which is larger than the
width of the averaging interval). The minimization consists in
the minimization of an averaged Hamiltonian.

This numerical technique has already been applied to the optimal
control of a low thrust space engine that must reach its final orbit;
the controls in the Keplerian case (where periodicity exists) and in the non Keplerian
case (where the influence of the moon and of the sun breaks the
periodicity) have been successfully computed \cite{goumri}.

The paper is organized as follows.
 Section \ref{settingssec} presents
the original OCP. It then presents windowed averaging for functions,
differential equations and control systems. It then presents the
concept of
averaged OCP that is used in this
paper. This is where the averaging error $\alpha$ is defined. Sections \ref{expansions}
and \ref{FormalExp} compute formal expansions of the state and of the costate of
the nominal problem with respect to the parameter $\alpha$.
Auxilliary variables that will be used later in the course of the
article are defined here. Section \ref{auxProblem} introduces an auxiliary problem of optimization as
well as new auxiliary variables. The main assumptions (bounded
derivatives and convexity) are given before we state the auxiliary
problem.
The main result is exposed in section \ref{MainTheoremSection}. It is
a
result on the control cost, trajectories and the optimal control.
 Section \ref{proof} is devoted to the proof of the main theorem.
 A conclusion is presented is section \ref{conclusionsec}.

 Note that, for a fluent reading of the paper,
 complicated computations are detailed in the appendices.

\section{The averaging method}\label{settingssec}
\subsection{Nominal Problem}\label{NominalProblem}
We wish to solve the following Optimal Control Problem (OCP) :
\begin{equation}
 \min_{u}\int_{0}^{T}L(x,u,t)dt
 \label{cost}
\end{equation}
where $x$ is is a finite dimensional state which satisfies the
dynamics :
\begin{equation}
 \frac{dx}{dt}=f(x,u,t) \; , \; x(0)=\chi_{0}
 \label{dynamics}
\end{equation}
where $u$ is an integrable finite dimensional and unconstrained control.
The assumptions on $f$ and $L$ are given in section \ref{Assumptions}.

\subsection{An example problem}\label{exampleProblem}
Let us consider a house in winter. It is cold outside of the house,
and the people inside of the house want to be warm. These people can
use a device to heat the house inside.

A simple representation of the system uses the following
\begin{itemize}
	\item the inputs are the outside temperature, and the heating
	power of the device

	\item the output is the temperature inside the house
\end{itemize}

It should be emphasized that the outer temperature here is a weather
forecast data, and that no tractable state space representation of the
weather evolution is available to the control designer. Weather forcasting
requires a global representation of the weather dynamics, and it must
be updated regularly with measures taken all around the globe. This
is not an easy problem.

Let us go back to our house heating system. A very simplified state
space representation of the house temperature evolution is the following

\begin{equation}
	\frac{d\theta}{dt}=a(\theta_{o}-\theta)+bu
	\label{houseModel}
\end{equation}
where $\theta$ is the temperature inside the house, $\theta_{o}$ is
the outside temperature and $u$ is the heating power.

The cost that should be minimized is typically
\begin{equation}
	J(u)=\int_{0}^{T}c(t)u(t) dt
	\label{houseProblemCost}
\end{equation}
where $c(t)$ is the time varying cost of the energy. Of course, the
problem makes sense only if we have some constraint on the state,
typically
\begin{equation}
	\theta_{\min}(t)\leq \theta(t)\leq \theta_{\max}(t)
	\label{houseConstraint}
\end{equation}
which represents the comfort requirement of the people inside. These
constraints vary in time because the house may be empty or not
depending on time.

The control is also bounded. A nice presentation of this problem in
context is in \cite{robillart}.

Observe that this problem is essentialy determined by three datasets:
\begin{itemize}
	\item the forecast for the outer temperature

	\item the evolution of the cost of the heating power

	\item the varying occupancy of the house.
\end{itemize}

In this article we will not elaborate in the solution of this
particular optimal control problem, notably because it involves state
constraints, and we shall focus on the behavior of the system (\ref{houseModel}).

This system features two time scales:

\begin{itemize}
	\item the time constant of the house is several days

	\item the sampling time for the outer temperature forcast data
	is one hour
\end{itemize}
Figure \ref{temperatureFig} shows the outer temperature data over two months.
\begin{figure}[htbp]
	\centering
	\includegraphics[width=0.8\linewidth]{Temperature.png}
	\caption{Outside temperature during two winter months. The x-axis
	is labeled in days, and the y-axis in degrees centigrade. Observe
	the fast behavior of the data.}
	\label{temperatureFig}
\end{figure}

The issue is that, while the time constant of the system is of
several days, a
discretization of the system should be performed with a time step of
one hour because of the fast variations of the outer temperature.
This is the problem that we shall address in this paper.

Before going on, let us recall a result on differential equations.
\subsection{An integral version of the Gronwall lemma}
\begin{lemm}\label{integralGronwallLemma}
	Let $x$ a real value function of time that satisfies the
	differential equation
	\begin{equation}
		\frac{dx}{dt}=f(x,t)\;\;x(0)=x_{0}
		\label{generalODE}
	\end{equation}
	and such that, for any integrable signal $z$ one has
	\begin{equation}
		\left|\int_{0}^{t}f(z(s),s)ds\right| \leq
		a\int_{0}^{t}|z(s)|ds+\left|\int_{0}^{t}u(s)ds\right|
		\label{growvallInequation}
	\end{equation}
	with $a\geq 0$, and let $y$ the (nonnegative) solution of
	\begin{equation}
		y(t) = | x_{0} | + a \int_{0}^{t}y(s)ds+ \left|
		\int_{0}^{t}u(s)ds\right|
\label{gronwallBounder}
	\end{equation}
	Then
	\begin{equation}
		|x(t)|\leq y(t)
		\label{gronwallBound}
	\end{equation}
\end{lemm}
\begin{proof}
	Using the Picard fixed point iterations, one checks that
	(\ref{gronwallBounder}) has a unique solution. We prove the lemma
	using also the Picard sequences for $x$ and $y$. The Picard iteration
	for $x$ is
	\begin{equation}
		x_{k+1}(t)=x_{0}+\int_{0}^{t}f(x_{k}(s),s)ds
		\label{Picardx}
	\end{equation}
	and for $y$:
	\begin{equation}
		y_{k+1}(t)=| x_{0} | + a \int_{0}^{t}y_{k}(s)ds+ \left|
		\int_{0}^{t}u(s)ds\right|
 \label{Picard}
	\end{equation}
	We have $y_{0}=|x_{0}|$, and, assuming that $|x_{k}|\leq y_{k}$,
	we derive
	\begin{eqnarray*}
		x_{k+1}(t) & \leq &
		|x_{0}|+a\int_{0}^{t}|x_{k}(s)|ds+\left|
		\int_{0}^{t}u(s)ds\right|
\\
		 & \leq & |x_{0}|+a\int_{0}^{t}y_{k}(s)ds+\left|
		\int_{0}^{t}u(s)ds\right|
\\
		 & = & y_{k+1}(t)
	\end{eqnarray*}
	and we obtain $|x|\leq y$ at the limit.
\end{proof}
Observe that we never use the differential formulation
(\ref{generalODE}) of the ODE in the proof.
\subsection{An integral estimate for perturbations}\label{bibo}
We use the previous lemma to estimate a kind of BIBO behavior using only the integral of the input.
\begin{theorem}\label{averagedPerturbation}
	Let $x$ the solution of (\ref{generalODE}) and $\lambda$ a
	Lipschitz constant of $f$. Let $g$ another vector field and $y$
	the solution of
\begin{equation}
			\frac{dy}{dt} = g(y,t)\;\;y(0)=x_{0}
	\label{perturbatedODE}
\end{equation}
	Define $\alpha$ by
	\begin{equation}
		\alpha= \max_{t\in [0,T]}
		\left|\int_{0}^{t}\left(f(y(s),s)-g(y(s),s)\right)ds\right|
		\label{integralErrorBound}
	\end{equation}
	Then, for $t\leq T$,
	\begin{equation}
		| x(t)-y(t)|\leq\alpha\frac{e^{\lambda T}-1}{\lambda}
		\label{introducingAlpha}
	\end{equation}
	\end{theorem}
\begin{proof}
	we have
	\begin{eqnarray*}
		|x(t)-y(t)| & = & \left|\int_{0}^{t}\left(f(x(s),s)-g(y(s),s)\right)ds\right|\\
		 & = &
		 \left|\int_{0}^{t}\left[f(x(s),s)-f(y(s),s)+f(y(s)s)-g(y(s),s)\right]ds\right|\\
		 & \leq & \lambda\int_{0}^{t}|x(s)-y(s)|ds +
		 \left|\int_{0}^{t}\left[f(y(s)s)-g(y(s),s)\right]ds\right|\\
		 & \leq & \lambda\int_{0}^{t}|y(s)-x(s)|ds +\alpha
	\end{eqnarray*}
	Using lemma \ref{integralGronwallLemma} and solving
	(\ref{gronwallBounder}) proves the result.
\end{proof}
\subsubsection{Discussion}
We assume that our nominal system is (\ref{generalODE}) and that the
basic solution methods for solving it are not very well conditioned.
This is the case, for instance, of our example house heating system.
We wish to solve the system (\ref{perturbatedODE}) instead. If we
{\em knew in advance} $y$, we could use some kind of signal
processing method on $g$ to obtain a better conditionned system with
a small error term $\alpha$. We cannot do this, however, because this
would amount to defining the vector fiel $g$ by using the solution
$y$ of (\ref{perturbatedODE}), which itself requires the definition
of $g$.

Instead we shall proceed as follows
\begin{itemize}
	\item First, we freeze the state $y$ in (\ref{integralErrorBound})
	to a constant value $\xi$. Obtaining a small $\alpha$ is then purely a
	problem of signal processing on known data. The method creates an
	approximate vectorfield $g=SP(f)$ that garantees that
	$\alpha$ is small for any $\xi$.

	\item We solve then (\ref{perturbatedODE}), and then compute the
	actual $\alpha$, i.e. the integral error (\ref{integralErrorBound}) computed using the trajectory $y(t)$.

	\item We derive an estimate of the distance between $y$ and $x$,
	{\em without ever solving (\ref{generalODE})}.
\end{itemize}

We can now proceed we the definition for averaging signals in next
section, and then use it to approximate differential systems in the
section thereafter.

\subsection{Filtering of a signal by windowed averaging}
We consider a signal defined on an interval and split this
interval into $N$ consecutive intervals. On each interval we compute
the average of the signal.
\begin{definition}\label{filtering}
Let $N$ an integer, and g an integrable function on $[0,T]$.
Let us subdivide $[0,T]$ into $N$ intervals $[t_{k},t_{k+1}]$, with:
\begin{equation}
 t_{k}=k\frac{T}{N} \; , \; k = 0..N
 \label{intervals}
\end{equation}
The averages of $g$ on these intervals define the low pass
filter $LP$ on $g$:
\begin{equation}
 LP[g](t) = \frac{1}{t_{k+1}-t_{k}}\int_{t_{k}}^{t_{k+1}}g(s)ds
 \; , \; t\in \left[t_{k},t_{k+1}\right) \; , \; k=0..N-1
 \label{LdefP}
\end{equation}
The difference $g-LP[g]$, that is the averaging error, defines the high pass filter $HP$ on $g$:
\begin{equation}
 HP=Id-LP
 \label{HPdef}
\end{equation}
We then denote I[g] the antiderivative of the averaging error:
\begin{equation}
 I[g](t)=\int_{0}^{t}HP[g](s)ds \; , \; t\in[0,T]
 \label{defI}
\end{equation}
\end{definition}
We show now that the upper bound of the function $I[g]$ is small when $N$ is big and
$g$ is bounded.
\begin{proposition}
 The following bound holds for any bounded function $g$ and any
 $t\in[0,T]$:
 \begin{equation}
 |I[g](t)| \leq 2 \|g\|_{\infty}\frac{T}{N}
 \label{BoundI}
\end{equation}
\end{proposition}
\begin{proof}
 Let us first prove that for any $k=0..N$
 \[ \int_{t_{k}}^{t_{k+1}}HP[g](t)dt = 0 \]
 Indeed:
\[
 \int_{t_{k}}^{t_{k+1}}HP[g](t)dt = \int_{t_{k}}^{t_{k+1}}g(t)dt
 -\int_{t_{k}}^{t_{k+1}}
 \left[\frac{1}{t_{k+1}-t_{k}}\int_{t_{k}}^{t_{k+1}}g(s)ds]\right]dt
 = 0
\]
 Hence, for $t\in[t_{k}t_{k+1}[$:
 \[ I[g](t) = \int_{t_{k}}^{t} HP[g](s)ds\]
 But $|LP[g](s)|\leq \|g\|_{\infty}$, so that $|HP[g](s)|\leq 2
 \|g\|_{\infty}$.
 This gives finally:
 \[ |I[g](t)|\leq 2 \|g\|_{\infty}(t-t_{n+1}) \leq 2
 \|g\|_{\infty}\frac{T}{N} \]
\end{proof}
Thus we may define a small number $\alpha$ that represents how close
the signal $g$ is closed to its averaged signal $LP[g]$:
\begin{equation}
	\alpha = \|I[g]\|_{\infty}
	\label{alphaSignal}
\end{equation}
\paragraph{Example 1}
Let us suppose that $g$ is periodical with period $\frac{T}{N}$.
Then $LP[g]$ is constant equal to the mean of $g$ over a period.
The function $HP[g]$ is a periodic signal with $0$ average.
The function $I[g]$ is the periodic antiderivative of $HP(g)$ with
value $0$ at $0$. Its upper bound $\alpha$ is of order $\frac{T}{N}$.
\paragraph{Example 2}
Let us suppose that $g$ is periodical with period $\epsilon$ a small
divisor of $T$, and $N=1$. Then $LP[g]$ is constant equal to the
mean of $g$ over the many periods in $[0,T]$.
The function $HP[g]$ is a periodic signal of small period $\epsilon$ with $0$ average.
The function $I[g]$ is the periodic antiderivative of $HP(g)$ with
value $0$ at $0$. Its upper bound $\alpha$ is of order $\epsilon$.
This example shows that $\alpha$ can be small with seemingly large
intervals, proving that (\ref{BoundI}) is a conservative estimate.
\paragraph{Application to the temperature signal}
We use this averaging method on the temperature signal of figure
\ref{temperatureFig} with an averaging window of 24 hours. Figure
\ref{averagingError} shows the difference between the signal and its
average; figure \ref{integrationError} shows the output of this
difference through an integrator. Figure \ref{integrationError}
indicates
that the averaging error $\alpha$ is much smaller than the sup norm
obtained from figure \ref{averagingError}.

\begin{figure}[htbp]
	\centering
	\includegraphics[width=0.8\linewidth]{DifferenceWithAverage.png}
 \caption{Difference between the temperature and its average.
		The maximum error value is 5.5914.}
		\label{averagingError}
\end{figure}
\begin{figure}[htbp]
	\centering
	\includegraphics[width=0.8\linewidth]{AveragingErrorThroughIntegrator.png}
 \caption{Output of the previous difference through an integrator. The
	 maximum error value is 0.0041}
		\label{integrationError}
\end{figure}

\subsection{Averaging of ordinary differential equation}
Let us consider the ordinary differential equation (ODE):
\begin{equation}
 \frac{dx}{dt}=f(x,t) \; , \; x(0)=\chi_{0}
 \label{ODE}
\end{equation}
with $f$ Lipschitz with respect to $x$ and integrable with respect to
t.

The averaged ODE is defined as:
\begin{equation}
 \frac{dx_{0}}{dt}=LP[f](x_{0},t) \; , \; x_{0}(0)=\chi_{0}
 \label{aveODE}
\end{equation}
where the low-pass filter $LP$ on a general function $g(x,t)$ is defined as:
\begin{equation}
 LP[g](\xi,t) = \frac{1}{t_{k+1}-t_{k}}
 \int_{t_{k}}^{t_{k+1}}g(\xi,s)ds
 \; , \; t \in [t_{k},t_{k+1}) \; , \; k=0..N-1
 \label{avedDynODE}
\end{equation}
As a summary, $LP[g]$ averages $g$ with respect to time on rectangular adjacent
windows, leaving the state variable unchanged.

The state $x_{0}$ is well defined and bounded because $f$, and thus
$LP[f]$, is Lipschitz.

We now estimate the distance between the trajectories obtained
respectively from the nominal and averaged system.
\begin{proposition}
 Let $x$ defined by the ODE (\ref{ODE}) and let $x_{0}$ defined by
 the averaged ODE (\ref{aveODE}).

	We define the signal $I[f,x_{0}]$ by
	 \begin{equation}
 I[f,x_{0}](t)=\int_{0}^{t}\left[f(x_{0}(s),s)-LP[f](x_{0}(s),s)\right]ds
 \label{IdynODE}
 \end{equation}

 Let $\lambda$ the Lipschitz constant of $f$ and define
	\begin{equation}
		\alpha = \sup_{t\in[0,T]}(|I[f,x_{0}](t)|)
		\label{alphaDefinitionODE}
	\end{equation}
 Then the following bound holds:
 \begin{equation}
 \|x-x_{0}\|_{\infty} \leq \alpha \frac{e^{\lambda
	T}-1}{\lambda}
	\label{alphaForODE}
 \end{equation}
\end{proposition}

In other words, the averaging error $ \|x-x_{0}\|_{\infty}$ is in
first order in $\alpha$, that is small when $N$ is big because
$f$ is Lipschitz and thus bounded on $[0,T]$.
\begin{proof}
	The result is proved by taking
	\[
		g(\xi,t)=LP[f](\xi,t)
	\]
	in Theorem \ref{averagedPerturbation}.
\end{proof}
If we want to have an {\em a priori} error estimate, i.e. an estimate
that does not depend on the trajectory $x_{0}$, we can use an error
bound linked to the difference between the vector fields.

\begin{proposition}[a priori error estimate]
	Define, for a state $\xi$,
	\begin{equation}
		\mathcal{I}[f](\xi,t)= \int_{0}^{t}\left[f(\xi,s)-LP[f](\xi,s)\right]ds
		\label{averaginErrorOnVectorFields}
	\end{equation}
	and
	\[
		A=\sup_{\xi,t}\left| \mathcal{I}[f](\xi,t) \right|
	\]
	Then
	\begin{equation}
		\alpha\leq A + \| f \|_{\infty}\left(\frac{T}{N}\right)^{2}
		\label{aprioriError}
	\end{equation}
\end{proposition}
\begin{proof}
	For $t\in \left[k\frac{T}{N},(k+1)\frac{T}{N}\right]$,
	\[
		\mathcal{I}[f](\xi,t)= \int_{k\frac{T}{N}}^{t}\left[f(\xi,s)-LP[f](\xi,s)\right]ds
	\]
	Let us use $\xi=x_{0}\left(k\frac{T}{N}\right)$. Since both $f$
	and $LP[f]$ are bounded by $\| f \|_{\infty}$, we derive that
\begin{eqnarray*}
	\alpha & \leq & A + 2\| f \|_{\infty}\int_{0}^{\frac{T}{N}}sds\\
	 & = & A +\| f \|_{\infty}\left(\frac{T}{N}\right)^{2}
\end{eqnarray*}
\end{proof}
\paragraph{Application to the example system}
Because the heating system is linear time invariant, we have $\alpha
= A$. For the simulation we take a time constant of $3$ days = $72$
hours. Figure \ref{trajectoryError} displays the superposition of the
inner temperature simulated with the original system (in black) and
the averaged system (in red). For this system we have $\alpha =0.2754$, and the
	maximum error of the output temperature is $0.4613$. Remember
	that the error estimate (\ref{alphaForODE}) is conservative since
	is does not take into account the stability of the system.

\begin{figure}[htbp]
	\centering
	\includegraphics[width=0.8\linewidth]{ComparisionBetweenTrajectories.png}
	\caption{Comparision of inner temperature between the original
	and averaged systems. For this system $\alpha =0.2754$, and the
	maximum error of the output temperature is $0.4613$.}
	\label{trajectoryError}
\end{figure}

\subsection{Averaging of a controlled system}

In the scope of this article, we define, for the integer $N>0$, and
any functions $u(t)$ and $g(x,u(t),t)$, an averaged function $LP[g,u]$
\begin{equation}\label{average}
	\begin{split}
 LP[g,u](x,t)= &
 \frac{N}{T}\int_{\frac{kT}{N}}^{\frac{(k+1)T}{N}}g(x,u(s),s)ds\\
 & \textrm{ for } t\in\left[
 \frac{kT}{N},\frac{(k+1)T}{N}\right) \textrm{ and } k\in[0,N-1]
	\end{split}
\end{equation}
As a summary, $LP[g]$ averages $g$ with respect to time (this
includes the open loop control) on rectangular adjacent windows,
leaving the state (or costate) unchanged. The control is considered
 as an input signal among others.

\subsection{Statement of the averaged optimal control problem}\label{aveProblem}
We define the averaged OCP which consists in minimizing the cost
\begin{equation}
 J_{0}(v)=\int_{0}^{T}LP[L,v](y(s),s)ds
 \label{averagecost}
\end{equation}
where $y$ is the state defined by
\begin{equation}
 \frac{dy}{dt}=LP[f,v](y,t)\; , \; y(0)=\xi_{0}
 \label{averagedyn}
\end{equation}
The system (\ref{averagedyn}) is a well defined differential equation.
\begin{assumption}\label{aveOCPsolution}
 The averaged OCP
 admits an optimal control $u_{0}$ with a corresponding trajectory
 $x_{0}$.
\end{assumption}
The optimal trajectory $x_{0}$ is defined by the ODE:
\begin{equation}
 \frac{dx_{0}}{dt}=LP[f,u_{0}](x_{0},t)\; , \; x_{0}(0)=\xi_{0}
 \label{aveOptTraj}
\end{equation}

Since this is a new kind of optimal control, it is wise to study its
stationarity condition.

\subsection{Stationarity condition for the averaged problem}
 \begin{theorem}
 Let H be the Hamiltonian of the nominal problem:
 \begin{equation}
 H(x,u,p,t) = L(x,u,t) + p \; f(x,u,t)
 \label{Hamiltonian}
 \end{equation}
 Let $p_{0}$ be the costate of the averaged problem, defined along the
 optimal trajectory $x_{0}$ by the backwards differential equation:
 \begin{equation}
 \frac{dp_{0}}{dt}=-LP\left[\frac{\partial H}{\partial x},u_{0}\right]
	 (x_{0},p_{0},t)\; , \; p_{0}(T)=0
 \label{aveCostate}
 \end{equation}
 Then the following stationarity condition holds:
 \begin{equation}
 \frac{\partial H}{\partial u}
	 (x_{0}(t),u_{0}(t),p_{0}(t),t)=0 \; a.e.
 \label{StatCond}
 \end{equation}
 where $a.e.$ stands for almost everywhere in $t \in [0,T]$.
\end{theorem}
\begin{proof}
 see appendix \ref{proofStationarity}
\end{proof}

\subsection{Introduction of a small $\alpha$}\label{alphaDefinition}
We define an averaging error $\alpha$ for the two point boundary
value problem similarly to what was defined for initial boundary
value problems.

For a function $g(x,u,t)$, the difference
$g(x_{0}(t),u_{0}(t),t)-LP]g,u_{0}](x_{0}(t),t)$ is the result of high pass
filtering $HP]g,u](x_{0}(t),t)$.
Let's define the antiderivative:
\begin{equation}
 I[g,x_{0},u_{0}](t)=\int_{0}^{t}HP[g,u_{0}](x_{0}(s),s)ds
 \label{Icontrol}
\end{equation}

Let's define similarly the backwards antiderivative
\begin{equation}
 I^{T}[g,x_{0},u_{0}](t)=\int_{t}^{T}HP[g,u_{0}](x_{0}(s),s)ds
 \label{ITcontrol}
\end{equation}

\begin{definition}\label{alphadef}
 For the rest of that document, we consider the small number $\alpha$
 defined as:
 \begin{equation}
 \alpha = \sup
 \left(
 \| I[f,x_{0},u_{0}]\|_{\infty} \; , \;
 \left\|I^{T}\left[\frac{\partial H}{\partial x},(x_{0},p_{0}),u_{0}\right]\right\| _{\infty}
 \right)
 \label{alpha}
 \end{equation}
\end{definition}

We shall see that $\alpha$ measures how close the nominal and
averaged problems are. Observe that $\alpha$ is small, for instance, when $N$ is big and assumption
\ref{boundedDeriv} below holds (bounded functions and their derivatives).

Observe that that $\alpha$ depends only on the solution of the
averaged problem. It is an a posteriori estimate like in definition
(\ref{alphaDefinitionODE}), and not an a priori estimate.

\subsection{Numerical perspective}\label{numericalSection}
Technically, the proof of our main approximation result is based on
the Pontryagin principle. This is a theoretical result which holds
independently of the actual numerical method used to solve the nominal (resp.
averaged) problem.

Now that we have stated what the averaged optimal control problem is,
we can begin to evaluate its benefit when solving numerically a
general
optimal control problem.

Let us consider a direct method for the solving of problem
(\ref{cost},\ref{dynamics}). To do so, we discretize the dynamics into
\begin{equation}
	x_{k+1}=x_{k}+\delta t \; f(x_{k},u_{k},t_{k})
	\label{discretized}
\end{equation}
In order to represent accurately the influence of the data on the
system, the discretization step $\delta t$ must be a suitable
sampling rate for the data. In this article's framework, $\delta t$ is
much smaller than the time constant of the system. The approximation
of the cost is obtained by using a quadrature rule for the integral.
Without using extra knowledge on the system, the time step of
the quadrature is $\delta t$.

In this framework, the optimal control problem becomes a finite
dimensional optimization problem with a finite number of equality
constraints.

We consider now a similar direct method to solve the averaged problem (\ref{averagecost},\ref{averagedyn}).
To avoid confusion, we denote by capital $T_{K}$ the boundaries of
the averaging intervals, and $X_{K}$ denotes the state of the
discretized averaged system. The dynamics of $X$ is
\begin{equation}
	X_{K+1}=X_{K}+\int_{T_{K}}^{T_{K+1}}f(X_{K},u(s),s)ds
	\label{discretizedAverage}
\end{equation}
where a suitable quadrature rule is used to compute the integral.
Consistently with the discretization (\ref{discretized}), we consider
the sampling rate of the control and data to be $\delta t$. The
computation of the cost is handled similarly.

We see that, while (\ref{discretized}) only consider the differential
aspect of the dynamics, the averaged dynamics
(\ref{discretizedAverage}) incorporates the integral action of a
dynamical system.

Let us now consider the benefit of solving the averaged problem versus
solving the nominal problem. To make things even simpler, let us use
a Riemann integral for the computation of (\ref{discretizedAverage})
and define
\[
R=\frac{T_{K+1}-T_{K}}{\delta t}
\]
We assume that $R$ is a (large) integer. Indeed, $\delta t$ is
dictated by the sampling rate of the data, while $T_{K+1}-T_{K}$ is
dictated by the time constant of the system, which is much larger.

We see that the number of control variables $u_{k}$ is the same in
both problems. However, the \emph{number of constraints}
(\ref{discretizedAverage}) for the averaged problem is $R$ times smaller than the number of
constraints (\ref{discretized}) for the original problem. This is the
main benefit of using averaging to solve the optimal control problem
with direct methods.

Finally, observe that the definition of the averaged costate
(\ref{aveCostate}) indicates what should be the two point boundary
problem used in an indirect method for solving the averaged problem.
In this case, the benefit
of using the averaging method is a coarser grid for the two point
boundary value problem.

\subsection{An important remark}
In what follows we shall take great
care to give explicit expressions for the various constants that are
used throughout the article to obtain error bounds. This is not for
the sake of displayed complicated expressions. This is for the
following reason: the parameter $\alpha$, though very important in
the scope of this work, is not the only one to influence the error
bounds. Typically, the magnitude of the dynamics, cost, and their
derivatives are also very important. In fact, if some part of the dynamics
is large, this means that we are close to the situation where we want
to use averaging while also having singular perturbations, and this a
difficult framework to work with.

So yes, the details of the
expressions are also
important, and the tedious computations are necessary to obtain this
kind of granularity in the expression of the final result. The
homogeneity of the expressions of the bounds with respect to the
numerical parameters of the various assumptions gives precious
insight on their physical meaning.

\section{A priori expansions and definitions of auxiliary
variables}\label{expansions}
We are now going to define auxiliary variables by means of formal
expansions with respect to $\alpha$. In what follows, $u_{0}$,
$x_{0}$ and $p_{0}$ denote the optimal control, state and costate of
the averaged problem as described in section \ref{aveProblem}.

This section has no intrinsic numerical purpose. However, the
exansions obtained here will be used to define intermediate variables
that will actually be used in the numerical part of the article.
\subsection{Notations}
 \begin{definition}
 We denote the following variables from the state $x$ (resp. $x_{0}$), the control
 $u$ (resp. $u_{0}$) and the costate $p$ ($p_{0})$:
 \begin{align*}
	\sigma = (x,u) \; , \; \sigma_{0} = (x_{0},u_{0}) \\
	w = (x,u,p) \; , \; w_{0} = (x_{0},u_{0},p_{0})
 \end{align*}
\end{definition}
where $p$ is the costate of the nominal problem defined by the ODE
with final condition:
 \begin{equation}
 \frac{dp}{dt}=-\frac{\partial H}{\partial x}
	 (x,u,p,t)\; , \; p(T)=0
 \label{Costate}
 \end{equation}

\begin{definition}
 We denote the derivatives up to second order of functions with
 respect to the variables
 $x$, $u$ or $\sigma $ with indexes, on the model:
 \begin{align*}
 f_{x} = \frac{\partial f}{\partial x} \\
	H_{uu} = \frac{\partial^{2} H}{\partial u^{2}}
 \end{align*}
 \[
 H_{\sigma \sigma} =
 \left[ \begin{array}
 {cc}

 H_{xx} & H_{xu} \\

 H_{ux} & H_{uu} \\

\end{array} \right]
 \]
 \end{definition}

 \subsection{Formal expansion in $\alpha$}
 The variable $\alpha$ defined by the equation (\ref{alpha}) is
 viewed as a
 small quantity\footnote{Actually it is not required that $\alpha$ be
 small. Its purpose is to give a one dimensional direction to a
 variation in a high dimensional space.}. In Appendix \ref{FormalExp}
 we compute a formal expansion of the state $x$ and the costate $p$
 at the first order with respect to $\alpha$.

 These formal expansions lead to the definition of auxiliary
 variables $\tilde{x_{1}}$ and $\tilde{p_{1}}$ as follows:
\begin{definition}\label{xpTilde1}
	We define the auxiliary variables $\tilde{x_{1}}$ and
	$\tilde{p_{1}}$ the following way:
	\begin{equation}\label{xtilde1}
		\alpha \tilde{x_{1}} = I[f,x_{0},u_{0}]
	\end{equation}
	and
		\begin{equation*}
		\alpha \tilde{p_{1}} = I^{T}[H_{x},(x_{0},p_{0}),u_{0}]
	\end{equation*}
\end{definition}

The error terms $\tilde{x_{1}}$ and $\tilde{p_{1}}$ correspond to the
deviation error due to the approximation of the dynamics with respect
to their fast oscillating behavior.

In definition \ref{deltaDefs} of
section \ref{auxVariables} we define deviations that are defined in
{\em magnitude}, by contrast to the sole averaging error (at the
optimum) which is considered here.

A consequence of the definition of $\alpha \tilde{x_{1}}$ in
definition \ref{xpTilde1} and $\alpha$ in definition \ref{alphadef} is that:
\begin{equation}
 \|\tilde{x_{1}}\|_{\infty} \leq 1
 \label{xtilde1le1}
\end{equation}

And a consequence of the definition of $\alpha \tilde{p_{1}}$ in
definition(\ref{xpTilde1}) and
$\alpha$ in definition \ref{alphadef} is that:
\begin{equation}
 \|\tilde{p_{1}}\|_{\infty} \leq 1
 \label{ptilde1le1}
\end{equation}

\section{Auxiliary Problem}\label{auxProblem}
We define an auxiliary problem which is similar to the study of the
second variation in \cite{Bry1}.
\subsection{Assumptions}\label{Assumptions}
We make two kinds of assumption on the problem data; the
first is related to the regularity of the functions involved, and the
second is a convexity assumption (Legendre-Clebsch) that is common in optimal control
\cite{Bry1,bensPert,chaplais}.
\begin{assumption}[Regularity]\label{boundedDeriv}
 The derivatives, up to the third order, of $f$ and $L$ with respect to $x$ and $u$ are
bounded by some $k>0$.
\end{assumption}
A consequence of that assumption is that the value $\alpha$ defined by
equation (\ref{alpha}) is well defined and small if $N$ is sufficiently
big (or if the problem is periodic of small period).

As a consequence, we can bound the second derivative of the dynamics
$f$:
\begin{proposition}
\begin{equation}
 \left|f_{\sigma \sigma}(X,U,t)\left[
 \begin{array}
 {c}

 Y \\

 V \\

\end{array}
 \right]^{2}\right| \leq k(|Y|+|V|)^{2}
 \label{fsigmasigma}
\end{equation}
for any $(X,U,Y,V)$.
\end{proposition}
\begin{proof}
\begin{eqnarray*}
 \left|f_{\sigma \sigma}(X,U,t)\left[
 \begin{array}
 {c}

 Y \\

 V \\

\end{array}
 \right]^{2}\right| &
	= & |f_{xx}(X,U,t)Y^{2}
	 + 2f_{xu}(X,U,t)YV + f_{uu}(X,U,t)V^{2}| \\
 & \leq & k(|Y|+|V|)^{2}
\end{eqnarray*}
\end{proof}

Moreover, as $p_{0}$ is the solution of the ODE with final condition (\ref{aveCostate}), it is
differentiable and thus continuous of the bounded interval $[0,T]$,
so that it is bounded. So that another consequence of the
assumption \ref{boundedDeriv} involves the derivatives of the
Hamiltonian:
\begin{proposition}
 The hamiltonian $H(x,u,p_{0},t)$ and its derivatives up to the
 third order in $u$ and $x$ are bounded by a constant K.
\end{proposition}
\begin{proof}
 Take $K=(1+\|p_{0}\|_{\infty}) k$.
\end{proof}
As a consequence, the Hessian of the Hamiltonian is bounded:
\begin{proposition}
\begin{equation}
 \left|H_{\sigma \sigma}(X,U,p_{0}t)\left[
 \begin{array}
 {c}

 Y \\

 V \\

\end{array}
 \right]^{2}\right| \leq K(|Y|+|V|)^{2}
 \label{HsigmasigmaEq}
\end{equation}
for any $(X,U,Y,V)$.
\end{proposition}
\begin{proof}
 Similar as for $f_{\sigma \sigma}$.
\end{proof}
\begin{assumption}[Strong Legendre-Clebsch condition \cite{Bry1}]\label{convexity}
 There exists $\beta > 0$ so that for any $(x,u)$, the following
 holds:
 \begin{equation}
 H_{uu}(x,u,p_{0},t) \ge \beta Id
 \label{beta}
 \end{equation}
 and
 \begin{equation}
 \left[H_{xx} - H_{xu}(H_{uu})^{-1}H_{ux}\right](x,u,p_{0},t)
	 \geq 0
 \label{positive}
 \end{equation}
\end{assumption}
The consequence of equation (\ref{beta}) is that $H_{uu}$ is
invertible and $\|H_{uu}^{-1}\|_{\infty} \le \frac{1}{\beta} $ on any
$(x,u,p_{0})$.

The consequence of equation (\ref{positive}) is :
\begin{proposition}\label{Hsigmasigma}
 \begin{equation}
 H_{\sigma \sigma}(x,u,p_{0},t) \ge 0
 \label{positive1}
 \end{equation}
\end{proposition}
\begin{proof}
	see Appendix \ref{HsigmasigmaProof}
 \end{proof}

\subsection{Auxiliary Problem Definition and Properties}\label{AuxProblemStatement}
\begin{definition}\label{secondVariation}
 Let's define the following notations:
 \[
 H_{0 \sigma \sigma} = H_{\sigma \sigma} (w_{0},t)
 =
 \left[\begin{array}
 {cc}

 H_{0xx} & H_{0xu} \\

 H_{0ux} & H_{0uu} \\

 \end{array}\right]
 \]
 and
 \[
 f_{0x}=f_{x}(\sigma_{0},t) \; , \; f_{0u}=f_{u}(\sigma_{0},t)
 \]
 Then we define the auxiliary problem as the linear quadratic OCP
 for the state y steered by the control v:
 \begin{eqnarray}
 \frac{dy}{dt} & = & f_{0x}[\tilde{x_{1}}+y] + f_{0u}v \; , \;
	y(0)=0
 \label{auxstate}\\
 J_{1}(v) & = & \int_{0}^{T} \left[\frac{1}{2} \left[
 \begin{array}
 {cc}

 y & v \\

 \end{array}
 \right]
	H_{0 \sigma \sigma} \left[
	\begin{array}
	{c}

	 y \\

	 v \\

	\end{array}
	\right]
	+ \tilde{p_{1}} \left(f_{0x}y + f_{0u}v\right)
	\right]dt
 \label{auxcost}
 \end{eqnarray}
\end{definition}
where $\tilde{x}_{1}$ and $\tilde{p}_{1}$ are defined in definition
\ref{xpTilde1} as the averaging errors on the dynamics of the optimal
state and costate at the optimal solution of the averaged problem
(as defined in section \ref{aveProblem}).

\begin{proposition}
 There exists an optimal cost $v_{1}$ for the auxiliary problem.
\end{proposition}
\begin{proof}
 It is a convex linear quadratic problem because $H_{0 \sigma \sigma}$ is
 non-negative (equation (\ref{positive1})).
\end{proof}
\begin{definition}
	We denote $v_{1}$ an optimal control of the auxialty problem, $y_{1}$ the trajectory corresponding to
	$v_{1}$ and $q_{1}$ the corresponding costate.
\end{definition}
\begin{proposition}
	$y_{1}$ follows the dynamics:
	\begin{equation}
	 \frac{dy_{1}}{dt} = f_{0x}(y_{1}+\tilde{x_{1}}) + f_{0u}v_{1} \; , \;
		y_{1}(0)=0
	 \label{optauxstate}
	\end{equation}
\end{proposition}
\begin{proof}
	This is the definition of the dynamics of the auxiliary
	problem.
\end{proof}
Moreover, we have:
\begin{proposition}
	The stationarity condition $\frac{\partial H_{1}}{\partial u}(y_{1},v_{1},p_{1})
	= 0$ may be written the following way:
	\begin{equation}
	 H_{0xu} y_{1} + H_{0uu} v_{1} +(\tilde{p_{1}} + q_{1}) f_{0u} = 0
	 \label{auxstationarity}
	\end{equation}
	Moreover, the costate $q_{1}$ of the auxiliary problem follows the
	dynamics:
	\begin{equation}
	 \frac{dq_{1}}{dt} = - H_{0xx} y_{1} - H_{0xu} v_{1} -
	 (\tilde{p_{1}} +q_{1}) f_{0x}
	 \; , \; q_{1}(T) = 0
	 \label{auxcostate}
	\end{equation}
\end{proposition}
\begin{proof}
	The hamiltonian of the auxiliary problem expands in:
\begin{equation}
	\label{auxhamiltonian}
	\begin{split}
 H_{1}(y,v,q) & = \frac{1}{2}\left(H_{0xx}y^{2} + 2 H_{0xu}yv +
	H_{0uu}v^{2}\right) \\
 & + \tilde{p_{1}}\left(f_{0x}y +f_{0u}v\right)
 +q_{1}\left[f_{0x}(y+\tilde{x_{1}}) + f_{0u}v\right]
 \end{split}
\end{equation}
\end{proof}

Finally, we have:
\begin{proposition}\label{AuxProblemBound}
 $y_{1}$, $v_{1}$ and $q_{1}$ are bounded by a constant $M$.
 \end{proposition}
\begin{proof}
 The auxiliary problem is smooth and convex.
\end{proof}

\subsection{More auxiliary variables and their upper
bounds}\label{auxVariables}

\begin{definition}\label{deltaDefs}
 For any $u \in \mathbf{L}_{[0,T]}^{2}$, $x$ is the trajectory of
 the nominal dynamics (\ref{dynamics}).

 Let's then define the following notations:
 \begin{eqnarray*}
 \delta x = x-x_{0} & & \tilde{x} = \delta x - \alpha \tilde{x_{1}}\\
 \delta u = u-u_{0} & & \tilde{u} = \delta u\\
 \delta \sigma = (\delta x,\delta u) & & \rho(\lambda,\mu) = \sigma_{0} + \lambda \mu \delta \sigma
 \end{eqnarray*}
 Where $(u_{0},x_{0})$ is a solution of the averaged problem and
 $\alpha \tilde{x_{1}}$ is $I[f,x_{0},u_{0}]$ (equation (\ref{xtilde1})).
\end{definition}
The $\delta$ variables are deviations in magnitude around the optimal
control $u_{0}$, after a correction due to the pure averaging error on the
various dynamics.
\begin{definition}
 We define the following variables:
\[
 r=\tilde{x} - \alpha y_{1} \; , \; v=\tilde{u} - \alpha v_{1}
\]
\[
Z[\lambda,\mu](t) = v +[H_{uu}^{-1}H_{ux}](\rho(\lambda,\mu),p_{0},t)(r+\alpha \tilde{x_{1}})
\]
\[
 z^{2} = \int_{0}^{1} \int_{0}^{1} \lambda
		\|Z[\lambda,\mu]\|_{L^{2}}^{2} d \lambda d \mu
\]
\end{definition}

Here $r$ and $v$ are variations with respect to the solution of the
averaged optimal control problem, with the extra corrective terms
\begin{itemize}
	\item the averaging error on the optimal dynamics {\em
	considered as a pure signal}

	\item the correction term that corresponds to the second
	variation in reference \cite{Bry1}, as defined in definition
	\ref{secondVariation}.
\end{itemize}

The quantity $z^{2}$ is repeatedly used in \cite{bensPert} to study
perturbations in optimal control; it has been used later in \cite{maamria}
in a similar manner.

We will bound $\|r\|_{\infty}^{2}$ and $\|v\|_{2}^{2}$ with respect to $z^{2}$ and
$\alpha^{2}$. Bounding then $z^{2}$ with respect to $\alpha^{2}$ will
lead to the main result.
\begin{definition}\label{krv12definition}
 We define the following constants:

 \begin{eqnarray*}
 k_{r1} & = & 4 k^{2}T e^{2k\left(1+\frac{K}{\beta}\right)}
 \\
 k_{r2} & = & \frac{\left(\frac{K}{\beta}+\frac{1}{2}(1+2M)^{2}
 \alpha\right)^{2}}{\left(1+\frac{K}{\beta}\right)^{2}}
 \left(e^{k\left(1+\frac{K}{\beta}\right)}-1\right)^{2}
 \\
 k_{v1} & = & 6\left( 2 + \frac{K^{2}T k_{r1}}{\beta^{2}}\right)
 \\
 k_{v2} & = &\frac{6K^{2}T(k_{r2}+1)}{\beta^{2}}
 \end{eqnarray*}

	with:

	- $k$ is introduced in assumption \ref{boundedDeriv} about
	the boundedness of the derivatives of $f(x,u,t)$ and $L(x,u,t)$.

	- $K=(1+\|p_{0}\|_{\infty}) k$

	- $M$ is the upper bound of the optimal trajectory of the
	auxiliary problem introduced in section \ref{AuxProblemStatement}
	(proposition \ref{AuxProblemBound}).

	- $\alpha$ is the (hopefully) small quantity defined in equation
	(\ref{alpha}).

	- $\beta$ is the convexity constant of $H_{uu}$ introduced in equation
	(\ref{beta}) inside of the Legendre-Clebsch conditions of
	assumption \ref{convexity}.

\end{definition}

Let us recall we we stand in the expansion of the state and of the
control. We have
\begin{eqnarray*}
	x & = & x_{0}+\alpha\tilde{x}_{1}+\alpha y_{1}+r \\
	u & = & u_{0}+\alpha v_{1}+v
\end{eqnarray*}

\begin{proposition}\label{vrinequalities}
 The following inequalities hold:
\begin{equation}
 \|r\|_{\infty}^{2} \leq k_{r1}z^{2} + k_{r2} \alpha^{2}
 \label{rinequality}
\end{equation}
and:
\begin{equation}\
 \|v\|_{2}^{2} \leq k_{v1} z^{2} + k_{v2} \alpha^{2}
 \label{vinequality}
\end{equation}
\end{proposition}
\begin{proof}
 see appendix \ref{vrinequalitiesproof}
\end{proof}
We now define an extra correction term that will be used furter on.
We will show that this correction term is also bounded by values
that depend only on $z$ and $\alpha$.
\begin{definition}
 $r_{1}$ is defined by the following dynamics:
 \begin{equation}
		\begin{split}
 \frac{dr_{1}}{dt} = &
	f(r_{1}+x_{0}+\alpha(\tilde{x_{1}}+y_{1}),v+u_{0}+\alpha
	v_{1},t)\\
	& - f(x_{0}+\alpha(\tilde{x_{1}}+y_{1}),u_{0}+\alpha v_{1},t)
	\\ r_{1}(0) & =0
	\end{split}
 \label{r1dynamics}
 \end{equation}
\end{definition}
The state $r_{1}$ is driven by the difference between the dynamics
with state $r_{1}$ and input $v$, and the nominal velocity of the
system for the value of the state and controls that include all the
previously introduced correction terms.

We will bound $\|r-r_{1}\|_{\infty}$ in $\alpha^{2}$ and $\|r_{1}\|_{\infty}^{2}$ in $z^{2}$ and
$\alpha^{2}$. As a consequence, we will ``anchor'' our error
estimates to the nominal term $r_{1}$.
 \begin{definition}\label{kr345definition}
 We define the following constants:
 \[
 k_{r5} = \frac{T}{2}(1+2M)^{2}(e^{kT}-1) \; , \;
 k_{r3} = 2 k_{r1} \; , \;
 k_{r4} = 2 (k_{r2} + k_{r5}^{2}\alpha^{2})
 \]
 with $k$, $K$ and $M$ as in definition \ref{krv12definition}.
 \end{definition}
\begin{proposition}\label{rr1r1inequalities}
 The following inequalities hold:
 \begin{equation}
 \|r-r_{1}\|_{\infty} \leq k_{r5} \alpha^{2}
 \label{rr1inequality}
 \end{equation}
 and
 \begin{equation}
 \|r_{1}\|_{\infty}^{2} \leq k_{r3}z^{2} + k_{r4} \alpha^{2}
 \label{r1inequality}
 \end{equation}
 \end{proposition}
\begin{proof}
 see appendix \ref{rr1r1inequalitiesproof}
\end{proof}

\section{Main theorem}\label{MainTheoremSection}
\begin{definition}\label{kJdefinition}
 We define the following constants:
 	\begin{eqnarray*}
	 k_{J1} & = &
	 4 \sqrt{3}KMTk_{r1}+\left(\frac{3k}{2}+KM\right)Tk_{r3}
	 +\left(\frac{3k}{2}+(2\sqrt{3}+1)KM\right)k_{v1}
	 \\
	 k_{J2} & = & 2\sqrt{3}KTk_{r5}+k(1+2M)+2KTM(M+1)
	 \\&&
	 +\frac{3k}{2}Tk_{r2} +\left(kM+4\sqrt{3}KM\right)Tk_{r4}
	 +\left(\frac{3k}{2}+(2\sqrt{3}+1)KM\right)k_{v2}
	 \\
	 k_{J0}&=&\left[e^{kT}-1 + \frac{K}{2k^{2}T}\left(e^{kT}-1\right)^{2}
 +\frac{k}{2}\alpha\right]
	\\
	k_{J}& =& k_{J2}+k_{J0}
	\\
	k_{x} & = & \sqrt{2\left(\frac{2k_{J}k_{r1}}{\beta} +k_{r2} + (M+1)^{2}\right)}
	\\
	k_{u} & = & \sqrt{2\left(\frac{2k_{J}k_{v1}}{\beta} +k_{v2} +
	T^{2}M^{2}\right)}
	\end{eqnarray*}
	where:

	- $k$, $K$, $M$, $\alpha$ and $\beta$ are as in definition \ref{krv12definition}.

	- $k_{r1}$, $k_{r2}$, $k_{v1}$ and $k_{r2}$ are defined in
	definition \ref{krv12definition}

	- $k_{r3}$, $k_{r4}$ and
	$k_{r5}$ are defined in definition \ref{kr345definition}.
\end{definition}
Note that these constants depend only of $k$, $K$, $M$, $\alpha$,
$\beta$, and the horizon $T$.
\begin{assumption}\label{mainAssumption}
\[
 \alpha \leq \frac{\beta}{2k_{J1}}
\]
\end{assumption}
\paragraph{Remark} This condition appears because $\alpha$ is a
numerical value determined by data, and not a parameter that tends to
zero. In fact, the previous assumption is always satisfied if we can
chose $\alpha$ arbitrarily small.
\begin{theorem}[Main Theorem]\label{MainTheorem}
 Considering the nominal problem in section \ref{NominalProblem},
 let $H(x,u,p,t)$ its Hamiltonian, and let $J^{*}= \inf_{u}J(u)$ be
 its infimum cost.

 Let $u_{0}$ a solution of the averaged problem
 described in section \ref{aveProblem} with its trajectory $x_{0}$. Such a solution exists
 by assumption \ref{aveOCPsolution}.

 Let $\alpha$ be the small quantity defined in equation
 (\ref{alpha}) and let $\beta$ be the constant introduced in
 assumption \ref{convexity}.

 Let the set of constants ($k_{J1}, k_{J}, k_{x}, k_{u})$
 introduced in definition \ref{kJdefinition}.

 Then, under the set of assumptions listed in section \ref{Assumptions}
 and the assumption \ref{mainAssumption}, the following
 inequalities hold:

 - the suboptimality of the real system commanded by $u_{0}$ is limited
 to:
 \begin{equation}
 J^{*} \leq J(u_{0}) \leq J^{*} + k_{J} \alpha^{2}
 \label{suboptimality}
 \end{equation}

 - any trajectory $x$ of the nominal problem
 for a $u$ better than $u_{0}$ ($J(u) \leq J(u_{0})$), is close to
 $(x_{0},u_{0})$, with:
 \begin{eqnarray}
 \|x-x_{0}\|_{\infty} & \leq & k_{x} \alpha
 \label{closeTrajectory}\\
 \|u-u_{0}\|_{2} & \leq & k_{u} \alpha
 \label{closeControl}
 \end{eqnarray}
\end{theorem}

\section{Proof of the main result}\label{proof}
\subsection{Proof Process}
To prove the main theorem, we proceed the following way.

The section \ref{JuLowerBoundSection} is devoted to the search of a
lower bound of any real cost $J(u)$ of the nominal problem. That
lower bound contains two integral terms that do not depend on $u$, a
term in $z^{2}$ that is the only one depending on $u$, and a term in
$\alpha^{2}$, that dos not depend on $u$ either.

The section \ref{Ju0UppererBoundSection} is devoted to the search of
an upper bound of the real cost $J(u_{0})$ of the nominal problem
controlled by $u_{0}$. That upper bound contains the same two integral terms
as in the lower bound of $J(u)$ and a term in $\alpha^{2}$.

Then the section \ref{ProofSuboptimality} uses the assumption $\alpha
\leq \frac{\beta}{k_{J1}}$ to obtain a lower bound of $J(u)$
independent of $u$, so that it is also a lower bound for $J^{*}$.
That lower bound is combined with the upper bound of $J(u_{0})$
to bound the suboptimality $J(u_{0})-J^{*}$ as a function of
$\alpha^{2}$, as stated in
the equation (\ref{suboptimality}) of the first
part of the main theorem \ref{MainTheorem}.

Then the section \ref{ProofSuboptimality} uses the stronger assumption $\alpha
\leq \frac{\beta}{2k_{J1}}$ to obtain, for any $u$ better than
$u_{0}$, i.e. so that $J(u) \leq J(u_{0})$, an upper bound of
$z^{2}$. With that bound of $z^{2}$, upper bounds for $\|x-x_{0}\|_{\infty}$ and $
\|u-u_{0}\|_{2}$ are found in the equations (\ref{closeTrajectory}) and (\ref{closeControl}) of
the second part of the main theorem, with the help of the definitions and bounds
of $r$ and $v$ in section \ref{auxVariables}.

\subsection{Lower bound on the real cost of the nominal problem}\label{JuLowerBoundSection}
\subsubsection{Expansion of the real cost $J(u)$}
We use an expansion with integral remainder to compute the variation
of the cost with respect to the nominal integral cost for the nominal system
being driven by $u_{0}$.
\begin{proposition}\label{expJuProp}
 The cost $J(u) = \int_{0}^{T} L(x,u,t) dt$ for any command $u \in \mathbf{L}_{[0,T]}^{2}$
 expands the following way:
 \begin{equation}
		\begin{split}
 J(u) = & \int_{0}^{T} L(x_{0},u_{0},t) dt
	+ \alpha \int_{0}^{T} H_{0x} \tilde{x_{1}} dt
	+ \int_{0}^{T} HP[H_{x},u_{0}](x_{0},p_{0},t) \tilde{x} dt
	\\
	& + \int_{0}^{T} \int_{0}^{1} \int_{0}^{1} \lambda H_{\sigma
	\sigma}(\rho(\lambda,\mu),p_{0},t)(\delta \sigma)^{2}
	d \lambda d \mu dt
	\end{split}
 \label{expJu}
 \end{equation}
\end{proposition}
\begin{proof}
 see appendix \ref{ProofExpJU}.
\end{proof}
We are going now to study carefully the various terms involved in the
expansion (\ref{expJu}).
\subsubsection{Lower Bound of the third term of the expansion of
$J(u)$ in equation (\ref{expJu})}
\begin{proposition}\label{J3LowerBoundProp}
 The following inequality holds:
	\begin{equation}
		\begin{split}
 		&	\int_{0}^{T} HP[H_{x},u_{0}](x_{0},p_{0},t) \tilde{x} dt
		\\ & \geq
	\alpha \int_{0}^{T} \tilde{p_{1}}[f_{0x}r_{1} + f_{0u}v] dt
	-\frac{3k}{2}(Tk_{r3}+k_{v1}) \alpha z^{2}
	\\
	& - \left[2KTk_{r5} + k(1+2M) + \frac{k}{2}(3(Tk_{r4}+k_{v2})
	+4T(1+2M)^{2}) \alpha\right]\alpha^{2}
	\end{split}
		\label{J3LowerBound}
	\end{equation}
\end{proposition}
\begin{proof}
 See appendix \ref{J3LowerBoundProof}.
\end{proof}

\subsubsection{Lower Bound of the fourth term of the expansion of
$J(u)$ in equation (\ref{expJu})}
\begin{proposition}\label{J4LowerBoundProp}
 The following inequality holds:
 \begin{equation}
		\begin{split}
 & \int_{0}^{T} \int_{0}^{1} \int_{0}^{1} \lambda H_{\sigma
	\sigma}(\rho(\lambda,\mu),p_{0},t)(\delta \sigma)^{2}
	d \lambda d \mu dt \\ & \geq
	-2KM \left[TM+\sqrt{3}(2Tk_{r2}+k_{v2}+T(M^{2}+2))\alpha\right] \alpha^{2}
 \\
 & + \left[\beta - 2\sqrt{3}KM(2Tk_{r1}+k_{v1})\alpha\right] z^{2}
 \\
	& + \alpha \int_{0}^{T} \tilde{p_{1}}\left[y_{1},v_{1}\right]
	H_{0 \sigma \sigma}
	\left[
	\begin{array}
	{c}
	 r+\alpha\tilde{x_{1}} \\
	 v \\
	\end{array}
	\right] dt
	\end{split}
	\label{J4LowerBound}
 \end{equation}
\end{proposition}
\begin{proof}
 see appendix \ref{J4LowerBoundProof}.
\end{proof}

\subsubsection{Bound in absolute value for the sum of the integral
terms of the right hand sides of equations (\ref{J3LowerBound}) and
(\ref{J4LowerBound})}
\begin{proposition}
 Let's define $R$ as the sum of the integral
terms of the right hand sides of equations (\ref{J3LowerBound}) and
(\ref{J4LowerBound}):
\[
 R = \alpha \int_{0}^{T} \tilde{p_{1}}
 \left[f_{0x}r_{1} + f_{0u}v
 + \left[y_{1},v_{1}\right]
	H_{0 \sigma \sigma}
	\left[
	\begin{array}
	{c}
	 r+\alpha\tilde{x_{1}} \\
	 v \\
	\end{array}
	\right]\right] dt
\]
Then the following inequality holds:
\begin{equation}
 |R| \leq KM[Tk_{r3} + k_{v1}]
 + [2KTM + (2KTMk_{r5} + kTMk_{r4} + kMk_{v2})\alpha]\alpha^{2}
 \label{RestBound}
\end{equation}
\end{proposition}
\begin{proof}
 see appendix \ref{JuLowerBoundProof}.
\end{proof}

\subsubsection{Lower Bound of the real cost $J(u)$}
We capitalize on the previous results to obtain a lower bound on $J(u)$
for any control $u$.
\begin{lemm}\label{JuLowerBoundLemma}
 A lower bound of the cost $J(u)$ of the nominal system for any
 control $u$ is given by:
 \begin{equation}
 J(u) \geq \int_{0}^{T} L(x_{0},u_{0},t) dt
	+ \alpha \int_{0}^{T} H_{0x} \tilde{x_{1}} dt
	+(\beta - k_{J1}\alpha)z^{2}
	- k_{J2}\alpha^{2}
 \label{JuLowerBound}
 \end{equation}
	where $k_{J1}$ and $k_{J2}$ are defined in definition
	\ref{kJdefinition}.
\end{lemm}
\begin{proof}
 This is a consequence of equations (\ref{J3LowerBound}),
 (\ref{J4LowerBound}) and (\ref{RestBound})
\end{proof}

\subsection{Upper bound on the cost of the nominal system controlled
by $u_{0}$}\label{Ju0UppererBoundSection}
As in the previous section, we first compute an expansion with
integral remainder of the cost; then we study each term of the
expansion to obtain a global bound in section \ref{upperBoundJ0}.
\subsubsection{Expansion of the cost $J(u_{0})$}
\begin{definition}
 Let $x^{0}$ be the trajectory of the nominal problem controlled
 by $u_{0}$. It is defined by the dynamics:
 \begin{equation}
 \frac{dx^{0}}{dt}=f(x^{0},u_{0},t) \; , \; x^{0}(0)=\chi_{0}
 \label{dynamicsu0}
\end{equation}
For that trajectory, we set the notations:
 \begin{eqnarray*}
 \delta x^{0} = x^{0}-x_{0} & & \tilde{x}^{0} = \delta x^{0} - \alpha \tilde{x_{1}}\\
 & \rho^{0}(\lambda,\mu) = x_{0} + \lambda \mu \; \delta x^{0} &
 \end{eqnarray*}
 Where $(u_{0},x_{0})$ is the solution of the averaged problem and
 $\alpha \tilde{x_{1}}$ is $I[f,x_{0},u_{0}]$ (equation (\ref{xtilde1})).
\end{definition}
\begin{proposition}\label{expJu0Prop}
 The cost $J(u_{0}) = \int_{0}^{T} L(x^{0},u_{0},t) dt$ for the
 optimal command $u_{0}$ of the averaged problem
 expands the following way:
 \begin{align}
 J(u_{0}) = \int_{0}^{T} L(x_{0},u_{0},t) dt
	+ \alpha \int_{0}^{T} H_{0x} \tilde{x_{1}} dt
	+ \int_{0}^{T} HP[H_{x},u_{0}](x_{0},p_{0},t) \tilde{x}^{0} dt
	\nonumber \\
	+ \int_{0}^{T} \int_{0}^{1} \int_{0}^{1}
	\lambda H_{xx}(\rho^{0}(\lambda,\mu),u_{0},p_{0},t)(\delta x^{0})^{2}
	d \lambda d \mu dt
 \label{expJu0}
 \end{align}
\end{proposition}
\begin{proof}
 It is a consequence of proposition \ref{expJuProp} with $u=u_{0}$,
 so that $\delta u = 0$.
\end{proof}

\subsubsection{Comparison of $x^{0}$ and $x_{0}$}
\begin{proposition}\label{deltax0prop}
 The following inequality hold for $\delta x^{0} = x^{0}-x_{0}$:
 \begin{equation}
 |\delta x^{0}| \leq \frac{\alpha}{kT} \left(e^{kT}-1\right)
 \label{deltax0}
 \end{equation}
\end{proposition}
\begin{proof}
 As $\delta x^{0} = x^{0}-x_{0}$, $x^{0}$ follows the dynamics
 (\ref{dynamicsu0}) and $x_{0}$ follows the averaged dynamics
 (\ref{aveOptTraj}), we have the following integral equation:
 \begin{eqnarray*}
 \delta x^{0}& = & \int_{0}^{t} [f(x^{0},u_{0},t) -
	 LP[f,u_{0}](x_{0},t)]dt \\
 & = & \int_{0}^{t} [f(x^{0},u_{0},t) - f(x_{0},u_{0},t)]dt
	 + \int_{0}^{t} [f(x_{0},u_{0},t) - LP[f,u_{0}](x_{0},t)]dt
 \end{eqnarray*}
 Thus the following inequality holds:
\[
 |\delta x^{0}| \leq
	 \|f_{x}\|_{\infty}\left(\int_{0}^{t}|\delta x^{0}|ds\right) + |I[f,x_{0},u_{0}](t)|
	 \leq kT \left(\int_{0}^{t}|\delta x^{0}|ds\right) +\alpha
\]
Equation (\ref{deltax0}) follows from Gronwall lemma.
\end{proof}

\subsubsection{Upper Bound of the third term of the development of
$J(u_{0})$ (\ref{expJu0})}
\begin{proposition}
 The following inequality holds:
\begin{equation}
 \left|\int_{0}^{T} HP[H_{x},u_{0}](x_{0},p_{0},t) \tilde{x}^{0} dt \right|
 \leq \left(e^{kT}-1+\frac{k}{2}\alpha\right)\alpha^{2}
 \label{J03bound}
\end{equation}
\end{proposition}
\begin{proof}
 By definition of $\tilde{x}^{0}$, we have
 \[
 \tilde{x}^{0} = x^{0} - x_{0} - \alpha \tilde{x_{1}},
 \]
 so that:
 \begin{eqnarray*}
 \frac{d \tilde{x}^{0}}{dt} & = & f(x_{0}+\delta x^{0},u_{0},t)
	 - LP[f,u_{0}](x_{0},t) + HP[f,u_{0}](x_{0},t)\\
 & = & f(x_{0}+\delta x^{0},u_{0},t) - f(x_{0},u_{0},t)
 \end{eqnarray*}
Thus, by Taylor expansion of
$f(x^{0},u_{0},t)$ with integral remainder, we have:
\[
 \frac{d \tilde{x}^{0}}{dt} = f_{x}(x_{0},u_{0},t) \delta
	 x^{0}
 + \int_{0}^{1} \int_{0}^{1}
 \lambda f_{xx}((x_{0}+ \lambda \mu (\delta
 x^{0}),u_{0},t)(\delta x^{0})^{2} d \lambda d \mu
\]
 Thus, thanks to proposition \ref{deltax0prop}, we have:
 \begin{equation}
 \left| \frac{d \tilde{x}^{0}}{dt}\right|
 \leq (e^{kT}-1)\alpha +\frac{k}{2}\alpha^{2}
 \label{dx0bound}
 \end{equation}
 But an integration by part, together with the fact that $
 HP[H_{x},u_{0}](x_{0},p_{0},t) =-\alpha \frac{d \tilde{p_{1}}}{dt}$
 and that $\tilde{x}^{0}(0)=\tilde{p_{1}}(T)=0$ leads to:
 \begin{equation}
 \int_{0}^{T} HP[H_{x},u_{0}](x_{0},p_{0},t) \tilde{x}^{0} dt
	 = \alpha \int_{0}^{T} \tilde{p_{1}}
	 \frac{d\tilde{x}^{0}}{dt}dt
 \label{intbypart0}
 \end{equation}
 Including equation (\ref{dx0bound}) and the fact that
 $\|\tilde{p_{1}}\|_{\infty} \leq 1$ into equation
 (\ref{dx0bound}) proves equation (\ref{J03bound})
 \end{proof}

\subsubsection{Upper Bound of the fourth term of the development of
$J(u_{0})$ (\ref{expJu0})}
\begin{proposition}
 The following inequality holds:
 \begin{equation}
 \left|\int_{0}^{T} \int_{0}^{1} \int_{0}^{1}
	\lambda H_{xx}(\rho^{0}(\lambda,\mu),u_{0},p_{0},t)(\delta x^{0})^{2}
	d \lambda d \mu dt\right|
	\leq \frac{K}{2k^{2}T}\left(e^{kT}-1\right)^{2}\alpha^{2}
 \label{J04bound}
 \end{equation}
\end{proposition}
\begin{proof}
 This is a consequence of the proposition \ref{deltax0prop}.
\end{proof}

\subsubsection{Upper Bound of $J(u_{0})$}\label{upperBoundJ0}
Inserting equations (\ref{J03bound}) and (\ref{J04bound}) into
equation (\ref{expJu0} leads to:
\begin{lemm} \label{Ju0UpperBoundLemma}
 An upper bound of the cost $J(u_{0})$ of the nominal system
 controlled by $u_{0}$ is given by:
 \begin{equation}
 J(u_{0}) \leq \int_{0}^{T} L(x_{0},u_{0},t) dt
	+ \alpha \int_{0}^{T} H_{0x} \tilde{x_{1}} dt
	+ k_{J0}\alpha^{2}
 \label{Ju0UpperBound}\\
 \end{equation}
 where $k_{J0}$ is defined in definition
	\ref{kJdefinition}.
\end{lemm}

\subsection{Proof of the main theorem}
The main theorem is proved in two steps. The first step compares the
costs to estimate the suboptimality of the real system controlled by
$u_{0}$. The second step compares the trajectories and controls
which outperform $u_{0}$, to estimate how close they are from the
trajectory and control driven by $u_{0}$.
\subsubsection{Comparison of the real cost controlled by $u_{0}$ and the
infimum cost of the real system}\label{ProofSuboptimality}
Let's now use the assumption $\alpha \leq \frac{\beta}{2k_{J1}}$ of
the first part of the Main Theorem \ref{MainTheorem} into the
equation (\ref{JuLowerBound}) of Lemma \ref{JuLowerBoundLemma}. The
term in $z^{2}$ is then non negative, and we get the lower bound
independent of $u$:
\[
 J(u) \geq \int_{0}^{T} L(x_{0},u_{0},t) dt
	+ \alpha \int_{0}^{T} H_{0x} \tilde{x_{1}} dt
	- k_{J2}\alpha^{2}
\]
As that lower bound holds for any $u$, it is also a lower bound for
the infimum cost $J^{*}= \inf_{u}J(u)$:
\[
 J^{*} \geq \int_{0}^{T} L(x_{0},u_{0},t) dt
	+ \alpha \int_{0}^{T} H_{0x} \tilde{x_{1}} dt
	- k_{J2}\alpha^{2}
\]
so that:
\[
 \int_{0}^{T} L(x_{0},u_{0},t) dt
	+ \alpha \int_{0}^{T} H_{0x} \tilde{x_{1}} dt
	\leq J^{*} + k_{J2}\alpha^{2}
\]
Inserting this equation into the
equation (\ref{Ju0UpperBound}) of Lemma \ref{Ju0UpperBoundLemma},
together with the fact that $J(u_{0}) \geq J^{*}$, by definition of
$J^{*}$, proves the suboptimality equation (\ref{suboptimality}) in he Main Theorem
\ref{MainTheorem}, since $k_{J}=k_{J0}+k_{J2}$.

\subsubsection{Comparison of the controls and trajectories with and
without $u=u_{0}$}\label{TrajControlComp}
Let's consider a control $u$ better than $u_{0}$, i.e. such that
$J(u) \leq J(u_{0})$.

Let's now use in a stronger manner the assumption $\alpha \leq \frac{\beta}{2k_{J1}}$ of
the second part of the Main Theorem \ref{MainTheorem} into the
equation (\ref{JuLowerBound}) of Lemma \ref{JuLowerBoundLemma}. The
coefficient $z^{2}$ is then lower than $\frac{\beta}{2}$, and we get the lower bound
dependent of $z^{2}$, that depends on $u$:
\[
 J(u) \geq \int_{0}^{T} L(x_{0},u_{0},t) dt
	+ \alpha \int_{0}^{T} H_{0x} \tilde{x_{1}} dt
	+\frac{\beta}{2}z^{2}
	- k_{J2}\alpha^{2}
\]
Thus, together with the
equation (\ref{Ju0UpperBound}) of Lemma \ref{Ju0UpperBoundLemma}, we have the list of inequalities:
\begin{eqnarray*}
 \int_{0}^{T} L(x_{0},u_{0},t) dt
	+ \alpha \int_{0}^{T} H_{0x} \tilde{x_{1}} dt
	+\frac{\beta}{2}z^{2}
	- k_{J2}\alpha^{2}
 & \leq & J(u) \leq J(u_{0}) \\
 & \leq & \int_{0}^{T} L(x_{0},u_{0},t) dt
	+ \alpha \int_{0}^{T} H_{0x} \tilde{x_{1}} dt
	+ k_{J0}\alpha^{2}
\end{eqnarray*}
So that, with the definition of $k_{J}=K_{J0}+k_{J2}$, we have
\begin{equation}
 z^{2} \leq \frac{2k_{J}}{\beta}\alpha^{2}
 \label{z2bound}
\end{equation}
On the other hand, by definition of $r$, we have:
\[
 x-x_{0} = r + \alpha(\tilde{x_{1}}+y_{1})
\]
so that, together with the equation (\ref{rinequality}):
\[
 \|x-x_{0}\|_{\infty}^{2} \leq 2 \left( \|r\|_{\infty}^{2} +
 \alpha^{2}(1+M)^{2} \right)
 \leq 2 \left(k_{r1}z^{2} +
 (k_{r2}(1+M)^{2}) \alpha^{2} \right)
\]
Introducing equation (\ref{z2bound}) into that equation leads to
equation (\ref{closeTrajectory}) of the second part
of the Main Theorem \ref{MainTheorem}.

Now let's consider the fact that, by definition of $v$:
\[
 u-u_{0} = v + \alpha v_{1}
\]
so that:
\[
 \|u-u_{0}\|_{2}^{2} \leq 2 \left( \|v\|_{2}^{2} +
 \alpha^{2}T^{2}M^{2} \right)
 \leq 2 \left(k_{v1}z^{2} +
 (k_{v2}T^{2}M)^{2} \alpha^{2} \right)
\]
Introducing equation (\ref{z2bound}) into that equation leads to
equation (\ref{closeControl}) of the second part
of the Main Theorem \ref{MainTheorem}.

\section{Conclusion}\label{conclusionsec}
We have shown that averaging on consecutive intervals is a flexible
and efficient
way to approximate signals while getting rid of rapidly varying
features. We have extended this method to ordinary
differential equations, control systems and optimal control problems.
This provides a very flexible framework for the approximation of
optimal control problems with rapidly varying features.

Indeed, if $\alpha$ is a measure of the averaging error on the
dynamics, we have shown that the level of suboptimality due to the
use of the solution of the averaged problem into the original control
system is bound by some $k\alpha^{2}$, where $k$ depends on very
general bounds on the dynamics and on the convexity of the system.

This makes it possible to use a coarser time mesh in the numerical computation
of the solution of the optimal control problem, as explained in
section \ref{numericalSection}.

However, this does not decrease the number of optimizations of the
Hamiltonian. This stresses, if
needed, the requirement for fast minization routines in this kind of
optimal control problem solving.

As a final note, we can observe that the control here is not
constrained. However, using an interior penalty approach as in
\cite{maamria} should be a reasonable method to generalize this
article's result to the constrained input case.

\appendix

\section{Proof of the stationarity condition of the averaged
problem}\label{proofStationarity}
\subsection{Averaged two boundaries problem}
Let $u_{0}$ be the optimal control of the averaged problem
(\ref{averagedyn}) and let $x_{0}$ be the corresponding trajectory.
$x_{0}$ is defined by the ODE with initial condition
(\ref{aveOptTraj}).
Let $p_{0}$ be the costate of the optimal trajectory, defined by the
ODE with final condition (\ref{aveCostate}), with $H$ the hamiltonian
(\ref{Hamiltonian}).

The system constituted of the of equations (\ref{aveOptTraj}) and
(\ref{aveCostate}) is a two boundaries problem. It is defined as the
two boundaries problem corresponding to the averaged optimal control
problem.

\subsection{First variation in the direction of $\delta u$}
 Let $\delta u \in \mathbf{L}_{[0,T]}^2$ a scalar square integrable
 function on $[0,T]$.

 Let $\epsilon > 0$ and let $u_{\epsilon} = u_{0} + \epsilon \delta u$
 the variation of $u_{0}$ in the direction of $\delta u$.

 Since $J_{0}(u_{0})=\min_{u}J_{0}(u)$, the following stationarity condition
 holds:
 \begin{equation}
 \forall \; \delta u \in \mathbf{L}_{[0,T]}^2,
	\left(\frac{dJ_{0}(u_{\epsilon})}{d \epsilon}\right)_{\epsilon=0} = 0
 \label{J0min}
 \end{equation}

 Let $x_{\epsilon}$ be the trajectory corresponding to $u_{\epsilon}$,
 defined by the dynamics equation \ref{averagedyn} with
 $v=u_{\epsilon}$:
 \begin{equation}
 \frac{dx_{\epsilon}}{dt}=LP[f,u_{\epsilon}](x_{\epsilon},t)\; , \;
 x_{\epsilon}(0)=\xi_{0}
 \label{aveVarTraj}
 \end{equation}

 Let $\delta x$ be the variation trajectory corresponding to the
 direction $\delta u$ given by
 $\delta x = \left(\frac{dx_{\epsilon}}{d \epsilon}\right)_{\epsilon =0}$.

\begin{lemm} \label{deltax}
 $\delta x$ respects the following dynamics function:
 \begin{equation}
	\frac{d(\delta x)}{dt} =
	LP\left[ \frac{\partial f}{\partial x}, u_{0}\right](x_{0},t) \delta x
 +LP\left[ \frac{\partial f}{\partial u} \delta u, u_{0}\right](x_{0},t)
	\label{VarDyn}
 \end{equation}
\end{lemm}
\begin{proof}
\[
 \frac{d(\delta x)}{dt} = \frac{d}{dt}\left(\frac{dx_{\epsilon}}{d \epsilon}\right)_{\epsilon =0}
 = \left(\frac{d}{d \epsilon}\frac{dx_{\epsilon}}{dt}\right)_{\epsilon =0}\\
 = \left(\frac{d}{d \epsilon}LP[f,u_{0}+\epsilon \delta u](x_{\epsilon},t)\right)_{\epsilon =0}
\]
 Let $k$ be so that $\left[t_{k},t_{k+1}\right))$. Then:
 \begin{eqnarray*}
 \frac{d(\delta x)}{dt} & = &
	\left(\frac{d}{d \epsilon}\left(\frac{1}{t_{k+1}-t_{k}}
	+\int_{t_{k}}^{t_{k+1}}f(x_{\epsilon}(t),u_{0}(s)+\epsilon \delta u(s),s)ds\right)\right)
	_{\epsilon =0}\\
 & = & \frac{1}{t_{k+1}-t_{k}}
	+ \int_{t_{k}}^{t_{k+1}}\frac{d}{d \epsilon}
	\left(f(x_{\epsilon}(t),u_{0}(s)+\epsilon \delta u(s),s)\right)_{\epsilon =0}ds \\
 & = & \frac{1}{t_{k+1}-t_{k}} \int_{t_{k}}^{t_{k+1}} \left(
	\frac{\partial f}{\partial x}(x_{0(t),u_{0}(s),s}\delta x(t)
	+\frac{\partial f}{\partial u}(x_{0(t),u_{0}(s),s}\delta u(s)
	\right) ds \\
 & = & LP\left[ \frac{\partial f}{\partial x},
	 u_{0}\right](x_{0},t) \delta x
	 + LP\left[ \frac{\partial f}{\partial u} \delta u, u_{0}\right](x_{0},t)
 \end{eqnarray*}
\end{proof}

\subsection{Proof of the stationarity result}
Let's make use of Equation (\ref{J0min}) in developing
$\left(\frac{dJ_{0}(u_{\epsilon})}{d \epsilon}\right)_{\epsilon=0}$
for a given $\delta u$.
\begin{lemm}
 The derivative at $0$ of $J_{0}(u_{\epsilon})$ in $\epsilon$ is
 related to the hamiltonian by the following equation:
 \begin{equation}
 \left(\frac{dJ_{0}(u_{\epsilon})}{d \epsilon}\right)_{\epsilon=0}
	 = \int_{0}^{T} LP\left[ \frac{\partial H}{\partial u} \delta
	 u, u_{0}\right](x_{0},p_{0},t) dt
 \label{J0hamiltonian}
 \end{equation}
\end{lemm}
\begin{proof}
 Let's use the equation (\ref{averagecost}) and then commute the
 differentiation and integration:
\[
 \left(\frac{dJ_{0}(u_{\epsilon})}{d \epsilon}\right)_{\epsilon=0}
 = \left( \frac{d}{d \epsilon}\left[ \int_{0}^{T}
 LP[L,u_{\epsilon}](x_{\epsilon},t)dt \right] \right)_{\epsilon=0}
 = \int_{0}^{T} \left(\frac{d}{d \epsilon}\left[
 LP[L,u_{\epsilon}](x_{\epsilon},t)\right]\right)_{\epsilon=0}dt
\]
 For any $k \in [0,N-1]$ and for any $t \in [t_{k},t_{k+1})$,
 we have the definition (\ref{average}) of $LP$:
 \[
 LP[L,u_{\epsilon}](x_{\epsilon},t)
 = \frac{1}{t_{k+1}-t_{k}}
 \int_{t_{k}}^{t_{k+1}}L(x_{\epsilon}(t),u_{\epsilon}(s),s)ds
 \]
 Thus if we commute again the integration and the differentiation:
 \[
 \left(\frac{d}{d \epsilon}\left[
 LP[L,u_{\epsilon}](x_{\epsilon},t)\right]\right)_{\epsilon=0}
 = \frac{1}{t_{k+1}-t_{k}}
 \int_{t_{k}}^{t_{k+1}}
 \left(\frac{dL(x_{\epsilon}(t),u_{\epsilon}(s),s)}{d \epsilon}\right)_{\epsilon=0} ds
 \]
 But by definition of $x_{\epsilon}$, $u_{\epsilon}$, $\delta x$ and $\delta u$, we have:
\[
 \left(\frac{dL(x_{\epsilon}(t),u_{\epsilon}(s),s)}{d \epsilon}\right)_{\epsilon=0}
 = \frac{\partial L}{\partial x}(x_{0}(t),u_{0}(s),s) \delta x(t)
 + \frac{\partial L}{\partial u}(x_{0}(t),u_{0}(s),s) \delta u(s)
\]
 Thus by averaging on $[t_{k},t_{k+1}]$, the result is ($LP$ is
 linear):
\[
 \left(\frac{d}{d \epsilon}\left[
 LP[L,u_{0}](x_{\epsilon},t)\right]\right)_{\epsilon=0}
 = LP\left[ \frac{\partial L}{\partial x},u_{0}\right](x_{0}(t),t) \delta x(t)
 + LP\left[ \frac{\partial L}{\partial u} \delta u,u_{0}\right](x_{0}(t),t)
\]
 But because of the dynamics (\ref{aveCostate}) of the averaged
 costate $p_{0}$, we have, with the definition
 (\ref{Hamiltonian}) of the Hamiltonian:
 \[
 LP\left[ \frac{\partial L}{\partial x},u_{0}\right](x_{0}(t),t)
	 = -\frac{dp_{0}}{dt} - p_{0} LP\left[ \frac{\partial f}{\partial x},u_{0}\right](x_{0}(t),t)
 \]
 Thus integrating $\left(\frac{d}{d \epsilon}\left[
	LP[L,u_{0}](x_{\epsilon},t)\right]\right)_{\epsilon=0}$
 between $0$ and $T$, we obtain:
 \begin{equation}
 \left(\frac{dJ_{0}(u_{\epsilon})}{d \epsilon}\right)_{\epsilon=0} =
 \int_{0}^{T} -\frac{dp_{0}}{dt} \delta x
 - \int_{0}^{T} p_{0} LP\left[ \frac{\partial f}{\partial x},u_{0}\right](x_{0}(t),t)
 \delta x(t) dt
 + \int_{0}^{T}
 LP\left[ \frac{\partial L}{\partial u} \delta u,u_{0}\right](x_{0}(t),t) dt
 \label{interm1}
 \end{equation}
 Let's make an integration by part for the first term of that
 equation:
 \[
 \int_{0}^{T} -\frac{dp_{0}}{dt} \delta x = -\left[p_{0} \delta x\right]_{0}^{T}
 + \int_{0}^{T} \frac{d(\delta x)}{dt} p_{0}
 \]
 The variation of $p_{0} \delta x$ between $0$ and $T$ is null
 because $\delta x(0) = 0$ and $p_{0}(T) =0$.
 Thus, with the dynamics of $\delta x$ given by the Lemma
 \ref{deltax}, the following holds:
\[
 \int_{0}^{T} -\frac{dp_{0}}{dt} \delta x =
 \int_{0}^{T} p_{0} LP\left[ \frac{\partial f}{\partial x}, u_{0}\right](x_{0},t) \delta x(t) dt
 + \int_{0}^{T} p_{0} LP\left[ \frac{\partial f}{\partial u} \delta u, u_{0}\right](x_{0},t) dt
\]
 Let's insert this equation in the first term of equation
 (\ref{interm1}).It results in:
 \[
 \left( \frac{dJ_{0}(u_{\epsilon})}{d \epsilon} \right)_{\epsilon=0} =
 \int_{0}^{T} \left[ LP\left[ \frac{\partial L}{\partial u}
 \delta u,u_{0}\right](x_{0},t)
 + p_{0} LP\left[ \frac{\partial f}{\partial u} \delta u, u_{0}\right](x_{0},t) \right] dt
 \]
 This proves the equation (\ref{J0hamiltonian}) by definition of
 the Hamiltonian.
\end{proof}
Let's now make use of equation (\ref{J0hamiltonian}). Let's first fix
$t$ and let $k$ be so that $t \in [T_{k},t_{k+1})$. Then we have:
\[
 LP \left[ \frac{\partial H}{\partial u} \delta u,u_{0} \right](x_{0},p_{0},t)
 = \frac{1}{t_{k+1}-t_{k}}
 \int_{t_{k}}^{t_{k+1}}\frac{\partial H}{\partial u}(x_{0}(t),u_{0}(s),p_{0}(t),t) \delta u(s) ds
\]
Let's specialize $\delta u$ as a ``needle variation'':
\[
 \delta u = \frac{t_{k+1}-t_{k}}{\eta} \mathbf{1}_{[t,t+\eta]} \delta v
\]
with $\eta > 0$ so that $t+\eta < t_{k+1}$ and $\delta v \in
\mathbf{L}_{[0,T]}^2$.

Then we have:
\[
 LP \left[ \frac{\partial H}{\partial u} \delta u,u_{0} \right](x_{0},p_{0},t)
 = \frac{1}{\eta}
 \int_{t}^{t+\eta} \frac{\partial H}{\partial
	 u}(x_{0}(t),u_{0}(s),p_{0}(t),t) \delta v(s) ds
 \begin{array}
 {c}
 a.e. \\
 \longrightarrow \\
 \eta \rightarrow 0
 \end{array}
 \frac{\partial H}{\partial u}(x_{0}(t),u_{0}(t),p_{0}(t),t) \delta v(t)
\]
More precisely, the limit is the value of the function at $t$
everywhere the function is continue, that is for any $t$ possibly
except for a countable number of ``jumps''. As any countable set is
negligible, the limit holds almost everywhere.

Thus, because of the equations (\ref{J0min}) and
(\ref{J0hamiltonian}), we have:
\[
 \int_{0}^{T} \frac{\partial H}{\partial u}(x_{0}(t),u_{0}(t),p_{0}(t),t) \delta v(t) =0
\]
and this is true for any $\delta v \in \mathbf{L}_{[0,T]}^2$.
This proves the stationnarity result:
\[
 \frac{\partial H}{\partial u}(x_{0}(t),u_{0}(t),p_{0}(t),t) = 0 \; a.e.
\]
\paragraph{Remark} To prove this result, we have used the fact that
the control is unconstrained. The extension of the main result to the
constrained case can be considered using the same penalty approach as
in \cite{maamria}. In this approach, the constrained problem is
approximated by a sequence of unconstrained problems, where the
stationarity condition holds.

\section{Formal expansions in $\alpha$}\label{FormalExp}
We formally expand the state, costate and control variables with
respect to $\alpha$. This step leads to the {\em definitions} of the various
terms which are used throughout the article to finally yield the
bounds stated in the main theorem.
\subsection{Expansion of the state $x$}
The state variable $x$ is the solution of the original dynamics
equation (\ref{dynamics}) driven by the control $u$. Let us define a
formal expansion of $x$ and $u$ with regards to $\alpha$:
\begin{equation}
 x = x_{0} + \alpha x_{1}
 \label{statedev}
\end{equation}
\begin{equation}
 u = u_{0} + \alpha u_{1}
 \label{controldev}
\end{equation}
Note that the redundant definitions of $x_{0}$ and $u_{0}$ are consistent, as will
be seen later.

We now separate the expansion terms $x_{0}$ and $x_{1}$ of the state
into low ($LP$) and high
($HP$) frequencies components:
\begin{equation}
 x_{0} = \bar{x_{0}} +\tilde{x_{0}} \; , \;
 x_{1} = \bar{x_{1}} +\tilde{x_{1}}
 \label{statedev1}
\end{equation}
Because of the equations (\ref{statedev}) and (\ref{statedev1}), and
because the derivative of the high frequency signal $\tilde{x_{1}}$
is in $\frac{1}{\alpha}$ \footnote{The integral of $HP$ is in
$\alpha$, so we consider that the derivative of $HP$ is formally in
$\frac{1}{\alpha}$} , the derivative of $x$ has the following
expansion in $\alpha$:
\begin{equation}
 \frac{dx}{dt} = \left[ \frac{d \bar{x_{0}}}{dt}
 + \frac{d \tilde{x_{0}}}{dt}
 + \alpha \frac{d \tilde{x_{1}}}{dt} \right]
 + \alpha \frac{d \bar{x_{1}}}{dt}
 \label{derivdev}
\end{equation}
But, because $x$ is the solution of the original dynamics
equation (\ref{dynamics}), using the formal expansions in $\alpha$ of $x$
and $u$, and thanks to assumption
\ref{boundedDeriv}, we have another expansion of $\frac{dx}{dt}$ at
the first order in $\alpha$:
\begin{equation}
 \frac{dx}{dt} = f(x_{0},u_{0},t)
 + \alpha (f_{u}(x_{0},u_{0},t)u_{1} + f_{x}(x_{0},u_{0},t)x_{1})
 \label{derivdev1}
\end{equation}
Consequently, identifying the zero order terms in equations
(\ref{derivdev}) and (\ref{derivdev1}), we obtain:
\[
 \frac{d \bar{x_{0}}}{dt}
 + \frac{d \tilde{x_{0}}}{dt}
 + \alpha \frac{d \tilde{x_{1}}}{dt} = f(x_{0},u_{0},t)
\]
But by definition of $\bar{x_{0}}$ as the low frequency part of
$x_{0}$, we have:
\[
 \frac{d \bar{x_{0}}}{dt} = LP[f,u_{0}](x_{0},t)
\]
Consequently, by definition of $HP$, we have:
\[
 \frac{d \tilde{x_{0}}}{dt}
 + \alpha \frac{d \tilde{x_{1}}}{dt} = HP[f,u_{0}](x_{0},t)
\]
with initial value $0$. Integration the previous equation yields
\[
 \tilde{x_{0}} + \alpha \tilde{x_{1}} = I[f,x_{0},u_{0}]
\]
But $I[f,\bar{x_{0}},u_{0}]$ is of order 1 in $\alpha$ and
$\tilde{x_{0}}$ is of order 0. Thus $\tilde{x_{0}}=0$, that gives
$x_{0}=\bar{x_{0}}$ the solution of the averaged problem. A
consequence of that is that both definitions of $x_{0}$ and $u_{0}$
are consistent.

Moreover, we have a definition of $\tilde{x_{1}}$ consistent with
definition \ref{xpTilde1}:
\begin{equation*}
 \alpha \tilde{x_{1}} = I[f,x_{0},u_{0}]
\end{equation*}
so that so that the derivative of $\tilde{x_{1}}$ is
$-\frac{1}{\alpha}HP[f_{x},u_{0}](x_{0},p_{0},t)$, that is in $\frac{1}{\alpha}$.

\subsection{Expansion of the costate $p$}\label{costateExpProof}
The costate variable $p$ is the solution of the costate dynamics
equation with ending condition (\ref{Costate}). We expand it at the
first order with respect to $\alpha$:
\begin{equation}
 p = p_{0} + \alpha p_{1}
 \label{costatedev}
\end{equation}
We then separate the expansion terms $p_{0}$ and $p_{1}$ into low ($LP$) and high
($HP$) frequencies components:
\begin{equation}
 p_{0} = \bar{p_{0}} +\tilde{p_{0}} \; , \;
 p_{1} = \bar{p_{1}} +\tilde{p_{1}}
 \label{costatedev1}
\end{equation}
Because of the equations (\ref{costatedev}) and (\ref{costatedev1}), and
because the derivative of the high frequency signal $\tilde{p_{1}}$
is in $\frac{1}{\alpha}$, the derivative of $p$ has the following
expansion in $\alpha$:
\begin{equation}
 \frac{dp}{dt} = \left[ \frac{d \bar{p_{0}}}{dt}
 + \frac{d \tilde{p_{0}}}{dt}
 + \alpha \frac{d \tilde{p_{1}}}{dt} \right]
 + \alpha \frac{d \bar{p_{1}}}{dt}
 \label{coderivdev}
\end{equation}
On the other hand, $p$ is the solution of the costate dynamics
equation with ending condition (\ref{Costate}). Moreover, we have
defined the developments in $\alpha$ of $x$, $p$ and $u$.

Thus we have another development of $\frac{dp}{dt}$ at the first order in $\alpha$:
\begin{equation}
 \frac{dp}{dt} = -H_{x}(x_{0},u_{0},p_{0},t)
 + \alpha (-H_{xu}(x_{0},u_{0},p_{0},t)u_{1}
 -H_{xx}(x_{0},u_{0},p_{0},t)x_{1} -f_{x}(x_{0},u_{0},t)p_{1})
 \label{coderivdev1}
\end{equation}
Consequently, identifying the zero order terms in equations
(\ref{coderivdev}) and (\ref{coderivdev1}), we have:
\[
 \frac{d \bar{p_{0}}}{dt}
 + \frac{d \tilde{p_{0}}}{dt}
 + \alpha \frac{d \tilde{p_{1}}}{dt} = -H_{x}(x_{0},u_{0},p_{0},t)
\]
But by definition of $\bar{p_{0}}$ as the low frequency part of
$p_{0}$, we have:
\[
 \frac{d \bar{p_{0}}}{dt} = -LP[H_{x},u_{0}](x_{0},p_{0},t)
\]
Consequently, by definition of $HP$, we have:
\[
 \frac{d \tilde{p_{0}}}{dt}
 + \alpha \frac{d \tilde{p_{1}}}{dt} = -HP[H_{x},u_{0}](x_{0},p_{0},t)
\]
with final value $0$. Integration of the previous equation yields
\[
 \tilde{p_{0}} + \alpha \tilde{p_{1}} = I^{T}[H_{x},(x_{0},p_{0}),u_{0}]
\]
But $I[-H_{x},x_{0},p_{0},u_{0}]$ is of order 1 in $\alpha$ as
$\tilde{p_{0}}$ is of order 0. Thus $\tilde{p_{0}}=0$, that gives
$p_{0}=\bar{p_{0}}$, the solution of the averaged costate equation.
A consequence of that is that both definitions of $p_{0}$ are consistent.

Moreover, we have a definition of $\tilde{p_{1}}$ consistent with
definition \ref{xpTilde1}:
\begin{equation*}
 \alpha \tilde{p_{1}} = I^{T}[H_{x},(x_{0},p_{0}),u_{0}]
 \label{ptilde1}
\end{equation*}
so that the derivative of $\tilde{p_{1}}$ is
$-\frac{1}{\alpha}HP[H_{x},u_{0}](x_{0},p_{0},t)$, that is in $\frac{1}{\alpha}$.

\section{Proof of non negativity of $H_{\sigma
\sigma}$}\label{HsigmasigmaProof}
 We make a proof by contradiction.

 Let's suppose that (\ref{positive1}) is not true. Then there
 exists a negative
 eigenvalue $- \gamma$ of $H_{\sigma \sigma}(x,u,p_{0},t)$, that is there
 exists an eigenvector $ \left[
 \begin{array}
 {c}

 y \\

 v \\

\end{array}
 \right]$ so that:
 \[
 H_{\sigma \sigma}(x,u,p_{0},t) \left[
 \begin{array}
 {c}

 y \\

 v \\

\end{array}
 \right]
 = - \gamma \left[
 \begin{array}
 {c}

 y \\

 v \\

\end{array}
 \right]
 \]
 This implies that the two following equations hold:
 \begin{eqnarray}
 H_{xx}(x,u,p_{0},t)y + H_{xu}(x,u,p_{0},t)v & = & - \gamma y
 \label{gammay}\\
 H_{ux}(x,u,p_{0},t)y + H_{uu}(x,u,p_{0},t)v & = & - \gamma v
 \label{gammav}
 \end{eqnarray}
 But $(\gamma Id + H_{uu)} \ge ( \gamma + \beta ) Id > 0 $, so that it
 is invertible and the equation (\ref{gammav}) can be solved in
 $v$, giving:
 \begin{equation}
 v = (\gamma Id + H_{uu}(x,u,p_{0},t))^{-1} H_{ux}y
 \label{vsolved}
 \end{equation}
 Then, replacing $v$ by its value in the equation (\ref{gammay}), we have:
 \begin{equation}
 (H_{xx}(x,u,p_{0},t) - H_{xu}(x,u,p_{0},t)(\gamma Id + H_{uu}(x,u,p_{0},t))^{-1} H_{ux}(x,u,p_{0},t)) y = -\gamma y
 \label{gammay1}
 \end{equation}
 But as $\gamma > 0$, we have the succession of inequalities:
\[
 (\gamma Id + H_{uu}(x,u,p_{0},t)) \ge H_{uu}(x,u,p_{0},t)
\]
	then
\[
 (\gamma Id + H_{uu}(x,u,p_{0},t))^{-1} \le H_{uu}^{-1}(x,u,p_{0},t)
\]
	and then
\begin{align*}
 (H_{xx}(x,u,p_{0},t) - H_{xu}(x,u,p_{0},t)(\gamma Id +
	 H_{uu}(x,u,p_{0},t))^{-1} H_{0ux}(x,u,p_{0},t)) \\
 \ge \left[(H_{xx}- H_{xu} H_{uu}^{-1} H_{ux})\right](x,u,p_{0},t) \ge 0
\end{align*}
 Thus the equation (\ref{gammay1}) can not hold, because $-\gamma <
 0$ can not be an eigenvalue.

The Proposition \ref{Hsigmasigma} is thus proved by contradiction.

\section{Proof of the inequalities on $r$, $v$, $r-r_{1}$ and $r_{1}$}
\subsection{Proof of proposition
\ref{vrinequalities}}\label{vrinequalitiesproof}
Proposition \ref{vrinequalities} is important since it bounds $r$ and
$v$ with respect to $z$ and $\alpha$. To obtain a bound on $r$ and
$v$ which only depends on $\alpha$, it will be enough to provide a bound of
$z$ involving $\alpha$ only.
\subsubsection{Upper bound for $r$}
\begin{proposition}
 The dynamics of $r$ is the following:
 \begin{equation}
 		 \label{rdynsmics}
		\begin{split}
 \frac{dr}{dt} & = f(r+x_{0}+ \alpha ( \tilde{x_{1}} + y_{1}), v + u_{0} + \alpha v_{1}, t)
 - f(x_{0}+ \alpha ( \tilde{x_{1}} + y_{1}), u_{0} + \alpha v_{1}, t)
 \\
 & + \alpha^{2} \int_{0}^{1} \int_{0}^{1}
		 \lambda f_{\sigma \sigma}(x_{0}+\lambda\mu\alpha(\tilde{x_{1}+y_{1}},
		 v + u_{0} + \alpha v_{1},t)
		 \left[
		 \begin{array}
		 {c}
		 \tilde{x_{1}} + y_{1} \\
		 v_{1}
		 \end{array}
		 \right]^{2} d \lambda d \mu
		\end{split}
\end{equation}
\end{proposition}
\begin{proof}
 By definition of $r$ and $v$, we have:
 \begin{eqnarray*}
 r=x-x_{0}-\alpha(\tilde{x_{1}}+y_{1}), & so \; that: & x=r+x_{0}+\alpha(\tilde{x_{1}}+y_{1})
	\\
 v=u-u_{0}-\alpha v_{1}, & so \; that: & u=v+u_{0}+\alpha v_{1}
 \end{eqnarray*}
 Moreover:
 \begin{equation}
		 \label{rdynsmics0}
		\begin{split}
 \frac{dr}{dt} = & \frac{dx}{dt} - \frac{dx_{0}}{dt}
	- \alpha\frac{d\tilde{x_{1}}}{dt} - \alpha \frac{dy_{1}}{dt}
 \\
 = &
	 f(r+x_{0}+\alpha(\tilde{x_{1}}+y_{1}),v+u_{0}+v_{1},t)-LP[f,u_{0}](x_{0},t)
 \\
 & -HP[f,u_{0}](x_{0},t) - \alpha [f_{0x}(y1+\tilde{x_{1}}) + f_{0u}v_{1}]
 \\
 = &
	 f(r+x_{0}+\alpha(\tilde{x_{1}}+y_{1}),v+u_{0}+v_{1},t)-f(x_{0},u_{0},t)
 \\
 & - \alpha [f_{0x}(y1+\tilde{x_{1}}) + f_{0u}v_{1}]
	\end{split}
 \end{equation}
But a Taylor expansion of $f(x_{0}+ \alpha ( \tilde{x_{1}} + y_{1}),
u_{0} + \alpha v_{1}, t)$ is so:
\begin{equation}
	\label{Taylor}
	\begin{split}
 & f(x_{0} ( \tilde{x_{1}} + y_{1}), u_{0} + \alpha v_{1}, t)
 = f(x_{0},u_{0},t) + \alpha [f_{0x}(y1+\tilde{x_{1}}) + f_{0u}v_{1}]
 \\
 & + \alpha^{2} \int_{0}^{1} \int_{0}^{1}
		 \lambda f_{\sigma \sigma}(x_{0}+\lambda\mu\alpha(\tilde{x_{1}+y_{1}},
		 v + u_{0} + \alpha v_{1},t)
		 \left[
		 \begin{array}
		 {c}
		 \tilde{x_{1}} + y_{1} \\
		 v_{1}
		 \end{array}
		 \right]^{2} d \lambda d \mu
			 \end{split}
\end{equation}
Introducing equation (\ref{Taylor}) in equation (\ref{rdynsmics0})
proves equation (\ref{rdynsmics}).
\end{proof}
\begin{proposition}
 The following inequality holds:
 \begin{equation}
 \|r\|_{\infty} \leq ke^{k\left(1+\frac{K}{\beta}\right)}
	 \left[\sqrt{T}\|Z(\lambda,\mu)\|_{2}
	 +T\left(\frac{K}{\beta}+\frac{1}{2}((1+2M)^{2} \alpha\right)\alpha \right]
 \label{rinftybound}
 \end{equation}
\end{proposition}
\begin{proof}
 Equation (\ref{fsigmasigma}) about the upper bound of $f_{\sigma
 \sigma}$ leads to:
 \begin{eqnarray*}
 \left|f_{\sigma \sigma}(x_{0}+\lambda\mu\alpha(\tilde{x_{1}+y_{1}},
		 v + u_{0} + \alpha v_{1},t)
		 \left[
		 \begin{array}
		 {c}

		 \tilde{x_{1}} + y_{1} \\

		 v_{1} \\

		 \end{array}
		 \right]^{2}\right|
		 & \leq & k(|\tilde{x_{1}} + y_{1}|+|v_{1}|)^{2}
		 \\
		 & \leq & k(|\tilde{x_{1}}| + |y_{1}|+|v_{1}|)^{2}
 \end{eqnarray*}
	Hence:
 \begin{equation}
 \left|f_{\sigma \sigma}(x_{0}+\lambda\mu\alpha(\tilde{x_{1}+y_{1}},
 v + u_{0} + \alpha v_{1},t)
 \left[
 \begin{array}
 {c}

 \tilde{x_{1}} + y_{1} \\

 v_{1} \\

 \end{array}
 \right]^{2}\right| \leq k (1+2M)^{2}
 \label{fsigmasigma1}
 \end{equation}
 Moreover, the function f is Lipschitz in $x$ and $u$ with
 Lipschitz constant the bound of the derivatives $k$, so that:
 \begin{equation}
 |f(r+x_{0}+ \alpha ( \tilde{x_{1}} + y_{1}), v + u_{0} + \alpha v_{1}, t)
 - f(x_{0}+ \alpha ( \tilde{x_{1}} + y_{1}), u_{0} + \alpha v_{1}, t)|
			 \leq k(|r|+|v|)
 \label{Lipschitz}
 \end{equation}
 The equations (\ref{fsigmasigma1}) and (\ref{Lipschitz}) in the
 equation (\ref{rdynsmics}) give, together with the fact that
 $r(0)=0$ the following inequality:
 \begin{equation}
 |r(t)| \leq \int_{0}^{t} (|r(s)|+|v(s)|)ds +
 \frac{kT}{2}(2M+1)^{2} \alpha^{2}
 \label{rbound0}
 \end{equation}
 But by definition of $Z[\lambda,\mu]$, we have:

\begin{equation}
 v = Z[\lambda,\mu]
 -[H_{uu}^{-1}H_{ux}](\rho(\lambda,\mu),p_{0},t)(r+\alpha
 \tilde{x_{1}})
 \label{vexpression}
\end{equation}
 so that:
 \[
 |v(s)| \leq |Z[\lambda,\mu](s)| +
 \|H_{uu}^{-1}\|_{\infty}\|H_{ux}\|_{\infty}(|r(s)| + \alpha
 |\tilde{x_{1}}(s)|)
 \]
 But the assumption \ref{convexity} proves that $H_{uu}$ is
 invertible and that $H_{uu}^{-1}$ is bounded by $\frac{1}{\beta}$,
 so that:
 \begin{equation}
 |v(s)| \leq |Z[\lambda,\mu](s)| + \frac{K}{\beta}(|r(s)| + \alpha)
 \label{vbound0}
 \end{equation}
 Including equation(\ref{vbound0}) in equation (\ref{rbound0})
 leads to:
 \begin{equation}
 |r(t)| \leq k \left(1+\frac{K}{\beta}\right)\int_{0}^{t} (|r(s)|)ds +
 \frac{kKT}{\beta}\alpha + \frac{kT}{2}(2M+1)^{2} \alpha^{2} +
 k\int_{0}^{T} |Z[\lambda,\mu](t)||dt
 \label{rbound1}
 \end{equation}
 But Cauchy property leads to:
 \begin{equation}
 \int_{0}^{T} |Z[\lambda,\mu](t)||dt \leq \sqrt{T}\|Z(\lambda,\mu)]\|_{2}
 \label{Cauchy}
 \end{equation}
 Including equation (\ref{Cauchy}) in equation (\ref{rbound1}) lead
 to:
 \begin{equation}
 |r(t)| \leq k
 \left(1+\frac{K}{\beta}\right)\int_{0}^{t} (|r(s)|)ds +
 k\sqrt{T}\|Z(\lambda,\mu)\|_{2} +
 kT\left[\frac{K}{\beta} + \frac{1}{2}(2M+1)^{2} \alpha\right]\alpha
 \label{rbound2}
 \end{equation}
 Equation (\ref{rbound2}), together with Gronwall lemma, proves
 equation (\ref{rinftybound}).
\end{proof}
\paragraph{Proof of Equation (\ref{rinequality})}

Let's take the square of equation (\ref{rinftybound}):
\[
 \|r\|_{\infty}^{2} \leq 2k^{2}e^{2k\left(1+\frac{K}{\beta}\right)}
	 \left[T\|Z(\lambda,\mu)\|_{2}^{2}
	 +T^{2}\left(\frac{K}{\beta}+\frac{1}{2}((1+2M)^{2}
	 \alpha\right)^{2} \alpha^{2} \right]
\]
Let's now multiply by $\lambda$ and integrate relatively to $\lambda$
and $\mu$ between $0$ and $1$:
\[
 \frac{1}{2}\|r\|_{\infty}^{2} \leq 2k^{2}e^{2k\left(1+\frac{K}{\beta}\right)}
	 \left[Tz^{2} + \frac{1}{2}T^{2}\left(\frac{K}{\beta}+\frac{1}{2}((1+2M)^{2}
	 \alpha\right)^{2} \alpha^{2} \right]
\]

Multiplying by 2 that equation proves Equation (\ref{rinequality}).

\subsubsection{Upper bound for $v$}
Let's apply the triangular inequality for the $\mathbf{L}_{2}$ norm
to the expression of $v$ (\ref{vexpression}):
\[
 \|v\|_{2} \leq \|Z[\lambda,\mu]\|_{2}
 +\|[H_{uu}^{-1}H_{ux}](\rho(\lambda,\mu),p_{0},t)(r
 +\alpha\tilde{x_{1}})\|_{2}
\]
But:
\[
 \|[H_{uu}^{-1}H_{ux}](\rho(\lambda,\mu),p_{0},t)(r
 +\alpha\tilde{x_{1}}\|_{2} \leq \sqrt{T} \|[H_{uu}^{-1}H_{ux}](\rho(\lambda,\mu),p_{0},t)(r
 +\alpha\tilde{x_{1}})\|_{\infty} \leq
	 \frac{K \sqrt{T}}{\beta}(\|r\|_{\infty} + \alpha
	 \|\tilde{x_{1}}\|_{\infty})
\]
Thus, taking the squares:
\[
 \|v\|_{2}^{2} \leq 3\left[ \|Z[\lambda,\mu]\|_{2}
 + \frac{K^{2}T}{\beta^{2}}(\|r\|_{\infty}^{2} + \alpha^{2}
 \|\tilde{x_{1}}\|_{\infty}^{2})\right]
\]
Let's multiply by $\lambda$ and integrate relatively to $\lambda$
and $\mu$ between $0$ and $1$:
\[
 \frac{1}{2}\|v\|_{2}^{2} \leq 3\left[ z^{2}
 + \frac{K^{2}T}{2\beta^{2}}(\|r\|_{\infty}^{2} + \alpha^{2}\right]
 \leq 3\left[\left(1 + \frac{K^{2}Tk_{r1}}{2\beta^{2}} \right)z^{2}
 + \frac{K^{2}T}{2\beta^{2}}(1+k_{r2})\alpha^{2}\right]
\]
Multiplying by 2 that equation proves Equation (\ref{vinequality}).

\subsection{Proof of proposition
\ref{rr1r1inequalities}}\label{rr1r1inequalitiesproof}

We now turn our attention to the variable $r$, which can be
approximated by the variable $r_{1}$, up to an error which is bounded
by a term that depends only on $z$ and $\alpha$.

The variable $r$ follows the dynamic (\ref{rdynsmics}) with $r(0)=0$, and $r_{1}$
follows the dynamic (\ref{r1dynamics}).

Thus $r-r_{1}$ follows the dynamics:
 \begin{eqnarray*}
 \frac{d(r-r_{1})}{dt} & = & f(r+x_{0}+ \alpha ( \tilde{x_{1}} + y_{1}), v + u_{0} + \alpha v_{1}, t)
 - f(r_{1}+x_{0}+ \alpha ( \tilde{x_{1}} + y_{1}), v + u_{0} + \alpha v_{1}, t)
 \\
 & & + \alpha^{2} \int_{0}^{1} \int_{0}^{1}
		 \lambda f_{\sigma \sigma}(x_{0}+\lambda\mu\alpha(\tilde{x_{1}+y_{1}},
		 v + u_{0} + \alpha v_{1},t)
		 \left[
		 \begin{array}
		 {c}

		 \tilde{x_{1}} + y_{1} \\

		 v_{1} \\

		 \end{array}
		 \right]^{2} d \lambda d \mu
 \end{eqnarray*}
 Thus, in a similar way than for the upper bound of $r$, the
 following inequality holds:
 \[
 |r(t)-r_{1}(t)| \leq k\int_{0}^{t} |r(s)-r_{1}(s)|ds +
 \frac{kT}{2}(2M+1)^{2} \alpha^{2}
 \]
 The equation (\ref{rr1inequality}) follows from Gronwall lemma.

 The equation (\ref{r1inequality}) is then the consequence of
 equations (\ref{rinequality}) and (\ref{rr1inequality}), together
 with:
 \[
 \|r_{1}\|_{\infty}^{2} \leq 2 (\|r\|_{\infty}^{2} +
 \|r-r_{1}\|_{\infty}^{2})
 \]

\section{Proof of the expansion of the real cost (proposition \ref{expJuProp})}\label{ProofExpJU}
\begin{proposition}\label{expLprop}
 $L(x,u,t)$ expands the following way:
 \begin{equation}
 L(x,u,t) = L(x_{0},u_{0},t) + L_{x}(x_{0},u_{0},t) \delta x
	 + L_{u}(x_{0},u_{0},t) \delta u
		 + \int_{0}^{1} \int_{0}^{1}
		 \lambda L_{\sigma \sigma}(\rho(\lambda,\mu),t)(\delta \sigma)^{2} d \lambda d \mu
 \label{expL}
 \end{equation}
\end{proposition}
\begin{proof}
 It is a Taylor expansion of $L(x,u,t)$ with integral remainder.
\end{proof}
\begin{proposition}\label{expdxtildeprop}
 The dynamics $\frac{d \tilde{x}}{dt}$ of $\tilde{x}$ expands the
 following way:

 \begin{equation}
 \frac{d \tilde{x}}{dt} = f_{x}(x_{0},u_{0},t) \delta x + f_{u}(x_{0},u_{0},t) \delta u
		 + \int_{0}^{1} \int_{0}^{1}
		 \lambda f_{\sigma \sigma}(\rho(\lambda,\mu),t)(\delta \sigma)^{2} d \lambda d \mu
 \label{expdxtilde}
 \end{equation}
\end{proposition}
\begin{proof}
 By definition of $\tilde{x}$, we have
 \[
 \tilde{x} = x - x_{0} - \alpha \tilde{x_{1}},
 \]
 so that:
 \begin{eqnarray*}
 \frac{d \tilde{x}}{dt} & = & f(x_{0}+\delta x,u_{0} + \delta u,t)
	 - LP[f,u_{0}](x_{0},t) + HP[f,u_{0}](x_{0},t)\\
 & = & f(x_{0}+\delta x,u_{0} + \delta u,t) - f(x_{0},u_{0},t)
 \end{eqnarray*}
The equation (\ref{expdxtilde}) follows by Taylor expansion of
$f(x,u,t)$ with integral remainder.
\end{proof}
\begin{proposition}\label{expL1prop}
 $L(x,u,t))$ rewrites the following way:
\begin{equation}
 L(x,u,t) = L(x_{0},u_{0},t) + H_{0x} (\tilde{x} +\alpha \tilde{x_{1}})
	 - p_{0} \frac{d \tilde{x}}{dt}
		 + \int_{0}^{1} \int_{0}^{1}
		 \lambda H_{\sigma \sigma}(\rho(\lambda,\mu),p_{0},t)(\delta \sigma)^{2} d \lambda d \mu
 \label{expL1}
 \end{equation}
\end{proposition}
\begin{proof}
 We have the following identities:
 \[
 \delta x = \tilde{x} + \alpha \tilde{x_{1}}
 \]
 and:
 \[
 	L_{x}(x_{0},u_{0},t)= H_{0x} - p_{0} f_{x}(x_{0},u_{0},t)
 \]
 Moreover, because of the stationary condition of the averaged
 problem, we have:
 \[
 	L_{u}(x_{0},u_{0},t) = - p_{0} f_{u}(x_{0},u_{0},t)
 \]
 Finally we change the integral remainder of the expansion of
 $L(x,u,t)$ in equation (\ref{expL}) with:
 \[
 L_{\sigma \sigma}(\rho(\lambda,\mu),t)
	= H_{\sigma \sigma}(\rho(\lambda,\mu),p_{0},t)
	 - p_{0} f_{\sigma \sigma}(\rho(\lambda,\mu),t)
 \]
 Thus equation (\ref{expL}) leads to:
 \begin{align*}
 L(x,u,t) = L(x_{0},u_{0},t + H_{0x} (\tilde{x} +\alpha \tilde{x_{1}})
	 - p_{0} \left[f_{x}(x_{0},u_{0},t) +
		 f_{u}(x_{0},u_{0},t)
		 \right. \\ \left.
		 + \int_{0}^{1} \int_{0}^{1}
		 \lambda f_{\sigma \sigma}(\rho(\lambda,\mu),t)](\delta \sigma)^{2} d \lambda d \mu\right]
		 + \int_{0}^{1} \int_{0}^{1}
		 \lambda H_{\sigma \sigma}(\rho(\lambda,\mu),p_{0},t)(\delta \sigma)^{2} d \lambda d \mu
 \label{expL1}
 \end{align*}
Inserting equation (\ref{expdxtilde}) in this equation proves
proposition \ref{expL1prop}.
\end{proof}
\begin{proposition}
 The following equality holds:
\begin{equation}
 \int_{0}^{T} -p_{0} \frac{d\tilde{x}}{dt}dt = -
	 \int_{0}^{T} LP[H_{x},u_{0}](x_{0},t) \tilde{x}dt
 \label{partintp0}
\end{equation}
\end{proposition}
\begin{proof}
 Let's make an integration by part:
 \[
 \int_{0}^{T} -p_{0} \frac{d\tilde{x}}{dt} dt
 = - [p_{0}\tilde{x}]_{0}^{T} + \int_{0}^{T}
	 \frac{dp_{0}}{dt} \tilde{x} dt
 \]
 This leads to the equation (\ref{partintp0}) because $\tilde{x}(0)=0$,
 $p_{0}(T)=0$ and:
 \[
 \frac{dp_{0}}{dt} = - LP[H_{x},u_{0}](x_{0},t)
 \]
\end{proof}
Now inserting the equation (\ref{partintp0}) in the equation
(\ref{expL1}) integrated between $0$ and $T$ leads to equation
(\ref{expJu}), which ends the proof of proposition \ref{expJuProp}.

\paragraph{Remark} The remaining appendix sections are devoted to
the obtention of a lower bound on $J(u)$. This is achieved by
obtaining a lower bound on each term of the expansion (\ref{expJu})
of $J(u)$. Only the complicated terms require an extensive study;
this study is
detailed in sections \ref{J3LowerBoundProof},
\ref{J4LowerBoundProof}, \ref{JuLowerBoundProof} and finally
\ref{RestBoundProof}.

\section{Proof of the lower bound of the third term (Proposition
\ref{J3LowerBoundProp})}\label{J3LowerBoundProof}
\begin{proposition}
 The following inequality holds:
\begin{equation}
 \int_{0}^{T} HP[H_{x},u_{0}](x_{0},p_{0},t) \tilde{x} dt
 \geq -(2KTk_{r5}+k(1+2M))\alpha^{2} +
 \alpha \int_{0}^{T} \tilde{p_{1}} \frac{dr_{1}}{dt} dt
 \label{J3LowerBound0}
\end{equation}
\end{proposition}
\begin{proof}
 By definition of $r$, we have:
 \[
 \tilde{x} = r + \alpha y_{1} = (r-r_{1}) + (r_{1} + \alpha y_{1})
 \]
 So that
 \[
 \int_{0}^{T} HP[H_{x},u_{0}](x_{0},p_{0},t) \tilde{x} dt
	= \int_{0}^{T} HP[H_{x},u_{0}](x_{0},p_{0},t) (r-r_{1}) dt
	+ \int_{0}^{T} HP[H_{x},u_{0}](x_{0},p_{0},t) (r_{1} + \alpha y_{1}) dt
 \]
 But because of equation (\ref{rr1inequality}), we have
 \[
 \left|\int_{0}^{T} HP[H_{x},u_{0}](x_{0},p_{0},t) (r-r_{1}) dt\right|
 \leq T\|HP[H_{x},u_{0}]\|_{\infty}k_{r5}\alpha^{2} \leq
 2KTk_{r5}\alpha^{2}
 \]
 Moreover, by definition of $\alpha \tilde{p_{1}}$, we have:
 \[
 HP[H_{x},u_{0}](x_{0},p_{0},t) = -\alpha \frac{d\tilde{p_{1}}}{dt}(t)
 \]
 So that the following inequality holds:
 \[
 \int_{0}^{T} HP[H_{x},u_{0}](x_{0},p_{0},t) \tilde{x} dt
	\geq -2KTk_{r5}\alpha^{2} - \alpha \int_{0}^{T}
	\frac{d\tilde{p_{1}}}{dt}(\tilde{x_{1}}+ \alpha y_{1}) dt
 \]
 If we make an integration by part and use the fact that
 $\tilde{p_{1}}(T)=\tilde{x_{1}}(0)=y_{1}(0)=0$, we get:
 \[
 \int_{0}^{T} HP[H_{x},u_{0}](x_{0},p_{0},t) \tilde{x} dt
	\geq -2KTk_{r5}\alpha^{2} +
	\alpha \int_{0}^{T}
	\tilde{p_{1}}\left(\frac{d\tilde{x_{1}}}{dt} + \alpha\frac{dy_{1}}{dt}\right) dt
 \]
 $y_{1}$ follows the dynamics (\ref{optauxstate}) and
 $\|\tilde{p_{1}}\|_{\infty} \leq 1$, so that:
 \[
 \left|\alpha^{2} \int_{0}^{T} \tilde{p_{1}}\frac{dy_{1}}{dt} dt\right|
 \leq k\alpha^{2} (\|y_{1}+\tilde{x_{1}}\|_{\infty} +\|v_{1}\|_{\infty}
	\leq k(2M+1) \alpha^{2}
 \]
 So that the inequality (\ref{J3LowerBound0}) is proved.
\end{proof}
\begin{proposition}
 The dynamics of $r_{1}$ expands the following way:
 \begin{equation}
	\label{r1dynamicsexp}
	\begin{split}
 \frac{dr_{1}}{dt} = & [f_{0x}r_{1} + f_{0u}v]
 \\
	& + \int_{0}^{1} \int_{0}^{1}
	\lambda f_{\sigma \sigma}(x_{0} + \lambda \mu (r_{1} + \alpha(\tilde{x_{1}}+y_{1})),
	u_{0} + \lambda \mu (v + \alpha v_{1}),t)
	\left[\begin{array}
	{c}
	 r_{1} + \alpha(\tilde{x_{1}}+y_{1}) \\
	 v + \alpha v_{1}
	\end{array}\right]^{2} d \lambda d \mu
	 \\
	&- \alpha^{2} \int_{0}^{1} \int_{0}^{1}
	\lambda f_{\sigma \sigma}(x_{0} + \lambda \mu\alpha(\tilde{x_{1}}+y_{1}),
	u_{0} + \lambda \mu \alpha v_{1},t)
	\left[\begin{array}
	{c}
	 \tilde{x_{1}}+y_{1} \\
	 \alpha v_{1}
	\end{array}\right]^{2} d \lambda d \mu
	\end{split}
 \end{equation}
\end{proposition}
\begin{proof}
 The dynamics of $r_{1}$ (equation (\ref{r1dynamics})) is the difference
 between the quantities $f(r_{1}+x_{0}+\alpha(x_{1}+y_{1}),v+u_{0}+\alpha v_{1},t)$ and
 $f(x_{0}+\alpha(x_{1}+y_{1}),u_{0}+\alpha v_{1},t)$.

 Let's make the Taylor expansions of these quantities at
 $(x_{0},u_{0})$:
\begin{eqnarray*}
 f(r_{1}+x_{0} & + & \alpha(x_{1}+y_{1}),v+u_{0}+\alpha v_{1},t) \\
 &=&f(x_{0},u_{0},t)\\
 && + f_{0x} (r_{1} + \alpha(x_{1}+y_{1})) + f_{0u}(v+\alpha v_{1})
 \\
 && + \int_{0}^{1} \int_{0}^{1}
	\lambda f_{\sigma \sigma}(x_{0} + \lambda \mu (r_{1} + \alpha(\tilde{x_{1}}+y_{1})),
	u_{0} + \lambda \mu (v + \alpha v_{1}),t)
	\left[\begin{array}
	{c}
	 r_{1} + \alpha(\tilde{x_{1}}+y_{1}) \\
	 v + \alpha v_{1}\\
	\end{array}\right]^{2} d \lambda d \mu
\end{eqnarray*}
and
\begin{eqnarray*}
 f(x_{0} & + & \alpha(x_{1}+y_{1}), u_{0} + \alpha v_{1},t) \\
 &=&f(x_{0},u_{0},t)\\
 && + \alpha [f_{0x}(x_{1}+y_{1})) + f_{0u}v_{1}]
 \\
 && + \alpha^{2} \int_{0}^{1} \int_{0}^{1}
	\lambda f_{\sigma \sigma}(x_{0} + \lambda \mu\alpha(\tilde{x_{1}}+y_{1}),
	u_{0} + \lambda \mu \alpha v_{1},t)
	\left[\begin{array}
	{c}
	 \tilde{x_{1}}+y_{1} \\
	 v_{1} \\
	\end{array}\right]^{2} d \lambda d \mu
\end{eqnarray*}
The simplifications between the two expansions while we take their
differences gives the dynamics of $r_{1}$ (\ref{r1dynamicsexp}).
\end{proof}
The consequence of that dynamics expansion is that:
\begin{equation}
 \label{Expansion2}
	\begin{split}
 & \alpha \int_{0}^{T} \tilde{p_{1}} \frac{dr_{1}}{dt}dt
 =\alpha \int_{0}^{T} \tilde{p_{1}}[f_{0x}r_{1} + f_{0u}v] dt
 \\
	&
 + \alpha \int_{0}^{T} \tilde{p_{1}}
 \left[\int_{0}^{1} \int_{0}^{1}
	\lambda f_{\sigma \sigma}(x_{0} + \lambda \mu (r_{1} + \alpha(\tilde{x_{1}}+y_{1})),
	u_{0} + \lambda \mu (v + \alpha v_{1}),t)
	\left[\begin{array}
	{c}
	 r_{1} + \alpha(\tilde{x_{1}}+y_{1}) \\
	 v + \alpha v_{1}
	\end{array}\right]^{2} d \lambda d \mu\right]dt
 \\ &
 - \alpha^{3} \int_{0}^{T} \tilde{p_{1}}
 \left[ \int_{0}^{1} \int_{0}^{1}
	\lambda f_{\sigma \sigma}(x_{0} + \lambda \mu\alpha(\tilde{x_{1}}+y_{1}),
	u_{0} + \lambda \mu \alpha v_{1},t)
	\left[\begin{array}
	{c}
	 \tilde{x_{1}}+y_{1} \\
	 v_{1}
	\end{array}\right]^{2} d \lambda d \mu \right]dt
	\end{split}
\end{equation}
\begin{proposition}
 Let's define the terms:
 \[
 R_{1} = \alpha \int_{0}^{T} \tilde{p_{1}}
 \left[\int_{0}^{1} \int_{0}^{1}
	\lambda f_{\sigma \sigma}(x_{0} + \lambda \mu (r_{1} + \alpha(\tilde{x_{1}}+y_{1})),
	u_{0} + \lambda \mu (v + \alpha v_{1}),t)
	\left[\begin{array}
	{c}
	 r_{1} + \alpha(\tilde{x_{1}}+y_{1}) \\
	 v + \alpha v_{1}\\
	\end{array}\right]^{2} d \lambda d \mu\right]dt
 \]
 and
 \[
 R_{2} = \alpha^{3} \int_{0}^{T} \tilde{p_{1}}
 \left[ \int_{0}^{1} \int_{0}^{1}
	\lambda f_{\sigma \sigma}(x_{0} + \lambda \mu\alpha(\tilde{x_{1}}+y_{1}),
	u_{0} + \lambda \mu \alpha v_{1},t)
	\left[\begin{array}
	{c}
	 \tilde{x_{1}}+y_{1} \\
	 v_{1} \\
	\end{array}\right]^{2} d \lambda d \mu \right]dt
 \]
 Then the following inequalities hold:
 \begin{equation}
 |R_{1}| \leq \frac{3k}{2}[(k_{r3}T + k_{v1}] \alpha z^{2}
	+[(k_{r4}T + k_{v2} + T(1+2M)^{2}] \alpha^{3}
 \label{R1bound}
 \end{equation}
 and
 \begin{equation}
 |R_{2}| \leq \frac{kT}{2}(1+2M)^{2} \alpha^{3}
 \label{R2bound}
 \end{equation}
\end{proposition}
\begin{proof}
 Equation (\ref{fsigmasigma}) about the upper bound of $f_{\sigma
 \sigma}$ leads to:
 \begin{eqnarray*}
 && \left|f_{\sigma \sigma}(x_{0} + \lambda \mu\alpha(\tilde{x_{1}}+y_{1}),
	u_{0} + \lambda \mu \alpha v_{1},t)
	\left[\begin{array}
	{c}
	 r_{1} + \alpha(\tilde{x_{1}}+y_{1} \\
	 v + \alpha v_{1} \\
	\end{array}\right]^{2}\right|\\
	&\leq & 3k[|r_{1}+\alpha(\tilde{x_{1}}+y_{1})| + |v+\alpha v_{1}|]^{2} \\
	&\leq & 3k[|r_{1}| + |v|+\alpha(|\tilde{x_{1}}+y_{1}| + |v_{1}|)]^{2} \\
	&\leq & 3k[r_{1}^{2} + v^{2}+\alpha(|\tilde{x_{1}}+y_{1}| + |v_{1}|)^{2}]
 \end{eqnarray*}
 Thus, together with the fact that $\tilde{p_{1}} \leq 1$, we have:
 \begin{eqnarray*}
 |R_{1}|& \leq &\frac{3k \alpha}{2}[\|r_{1}\|_{2}^{2} + \|v\|_{2}^{2}+ \alpha^{2}T(2M+1)^{2}]
 \\ |R_{1}| & \leq &\frac{3k \alpha}{2}[\|r_{1}\|_{\infty}^{2} + \|v\|_{2}^{2}+ \alpha^{2}T(2M+1)^{2}]
 \end{eqnarray*}
 Equations (\ref{r1inequality}) and (\ref{vinequality}) then lead
 to equation (\ref{R1bound}).

 Equation (\ref{fsigmasigma}) about the upper bound of $f_{\sigma
 \sigma}$ leads also to:

\[
 \left|
 f_{\sigma \sigma}(x_{0} + \lambda \mu\alpha(\tilde{x_{1}}+y_{1}),
 u_{0} + \lambda \mu \alpha v_{1},t)
 \left[\begin{array}
 {c}
 r_{1} + \alpha(\tilde{x_{1}}+y_{1} \\
 v + \alpha v_{1} \\
 \end{array}\right]^{2}\right|
 \leq k(|\tilde{x_{1}}+y_{1}| + |v_{1}|)^{2} \leq k(1+2M)^{2}
\]
 That proves equation (\ref{R2bound}) by triple integration and
 multiplication by $\alpha^{3}$.
 \end{proof}
The equations (\ref{R1bound}) and (\ref{R2bound}) included in
equation (\ref{R2bound}) (\ref{Expansion2}) lead to:
\begin{eqnarray*}
 \alpha \int_{0}^{T} \tilde{p_{1}} \frac{dr_{1}}{dt}dt
 & = &\alpha \int_{0}^{T} \tilde{p_{1}}[f_{0x}r_{1} + f_{0u}v] dt \\
 && -\frac{3k}{2}(k_{r2}T + k_{v1})Ma z^{2}
 - \frac{k}{2}[3(k_{r4}T+k_{v2}) +4T(1+2M)^{2}]\alpha^{3}
\end{eqnarray*}
Introducing that equation into equation (\ref{J3LowerBound0}) leads
to equation (\ref{J3LowerBound}) and thus the proposition
\ref{J3LowerBoundProp}.

\section{Proof of the lower bound of the fourth term (Proposition
\ref{J4LowerBoundProp})}\label{J4LowerBoundProof}
\begin{proposition}\label{J4LowerBound0Prop}
 The following inequality holds:
\begin{equation}
	\label{J4LowerBound0}
	\begin{split}
 &\int_{0}^{T} \int_{0}^{1} \int_{0}^{1} \lambda H_{\sigma
 \sigma}(\rho(\lambda,\mu),p_{0},t)(\delta \sigma)^{2}
 d \lambda d \mu dt \\
		&\geq
 -2KTM^{2} \alpha^{2} + \beta z^{2}
	\\
 &+ 2\alpha \int_{0}^{T} \int_{0}^{1} \int_{0}^{1} \lambda \left[y_{1},v_{1}\right]
	H_{\sigma \sigma}(\rho(\lambda,\mu),p_{0}t)
 \left[
 \begin{array}
 {c}
 r+\alpha\tilde{x_{1}} \\
 v \\
 \end{array}
 \right] d \lambda d \mu dt
 		\end{split}
\end{equation}
\end{proposition}
\begin{proof}
 Let's use the fact that:
 \[
 \delta \sigma =
	\left[
 \begin{array}
 {c}
 r+\alpha\tilde{x_{1}} \\
 v \\
 \end{array}
 \right]
	+\alpha
	\left[
 \begin{array}
 {c}
 y_{1} \\
 v_{1} \\
 \end{array}
 \right]
 \]
 to expand $H_{\sigma \sigma}(\rho(\lambda,\mu),p_{0},t)(\delta \sigma)^{2}$:
 \begin{equation}
 \label{expJ4}
	 \begin{split}
 H_{\sigma
 \sigma}(\rho(\lambda,\mu),p_{0},t)(\delta \sigma)^{2} & =
	H_{\sigma \sigma}(\rho(\lambda,\mu),p_{0},t)
	\left[
 \begin{array}
 {c}
 r+\alpha\tilde{x_{1}} \\
 v
 \end{array}
 \right]^{2}\\ &
	+\alpha^{2}
	H_{\sigma \sigma}(\rho(\lambda,\mu),p_{0},t)
		\left[
 \begin{array}
 {c}
 y_{1} \\
 v_{1}
 \end{array}
 \right]^{2}
 \\
 & + 2 \alpha [y_{1},v_{1}]
	H_{\sigma \sigma}(\rho(\lambda,\mu),p_{0},t)
 \left[
 \begin{array}
 {c}
 r+\alpha\tilde{x_{1}} \\
 v
 \end{array}
 \right]
		\end{split}
 \end{equation}
 The second term is easily upper bounded in absolute value
 (equation (\ref{HsigmasigmaEq}) about the upper bound of
 $H_{\sigma\sigma}$):
 \[
 \left|H_{\sigma \sigma}(\rho(\lambda,\mu),p_{0},t)
		\left[
 \begin{array}
 {c}
 y_{1} \\
 v_{1} \\
 \end{array}
 \right]^{2}\right|
	\leq K(|y_{1}|+|v_{1}|)^{2}
	\leq 4KM^{2}
 \]
 so that:
 \begin{equation}
 \left| \int_{0}^{T} \int_{0}^{1} \int_{0}^{1}
	H_{\sigma \sigma}(\rho(\lambda,\mu),p_{0},t)
		\left[
 \begin{array}
 {c}
 y_{1} \\
 v_{1} \\
 \end{array}
 \right]^{2}\right|
	d \lambda d \mu dt
	\leq 2KTM^{2}
 \label{J42bound}
 \end{equation}
 To upper bound the first term, let's expand it in its components:
 \begin{equation}
 \label{expJ41}
	 \begin{split}
 	H_{\sigma \sigma}(\rho(\lambda,\mu),p_{0},t)
	\left[
 \begin{array}
 {c}
 r+\alpha\tilde{x_{1}} \\
 v
 \end{array}
 \right]^{2}
	&=H_{xx}(\rho(\lambda,\mu),p_{0},t)(r+\alpha\tilde{x_{1}})^{2}\\
	&+2(r+\alpha\tilde{x_{1}})H_{xu}(\rho(\lambda,\mu),p_{0},t)v
	\\
	&+H_{uu}(\rho(\lambda,\mu),p_{0},t)v^{2}
	\end{split}
 \end{equation}
 Now let's use the definition of
 $Z[\lambda,\mu](t) = v +[H_{uu}^{-1}H_{ux}](\rho(\lambda,\mu),p_{0},t)(r+\alpha \tilde{x_{1}}$
 to expand the terms in $v$ and $v^{2}$:
 \begin{equation}
		 \label{expv}
		\begin{split}
 2(r+\alpha\tilde{x_{1}})H_{xu}(\rho(\lambda,\mu),p_{0},t)v & =
	2(r+\alpha\tilde{x_{1}})H_{xu}(\rho(\lambda,\mu),p_{0},t)Z[\lambda,\mu]
 \\ &
	- 2[H_{ux}H_{uu}^{-1}H_{xu}](\rho(\lambda,\mu),p_{0},t)(r+\alpha\tilde{x_{1}})^{2}
	\end{split}
 \end{equation}
 and
 \begin{equation}
		\label{expv2}
		\begin{split}
 H_{uu}(\rho(\lambda,\mu),p_{0},t)v^{2} & =
 H_{uu}(\rho(\lambda,\mu),p_{0},t)Z[\lambda,\mu]^{2}
 \\
 & + [H_{ux}H_{uu}^{-1}H_{xu}](\rho(\lambda,\mu),p_{0},t)(r+\alpha\tilde{x_{1}})^{2}
 \\
	& - 2(r+\alpha\tilde{x_{1}})H_{xu}(\rho(\lambda,\mu),p_{0},t)Z[\lambda,\mu]
	\end{split}
 \end{equation}
 Introducing equations (\ref{expv}) and (\ref{expv2}) in equation
 (\ref{expJ41}) leads to:
 \begin{eqnarray*}
 	H_{\sigma \sigma}(\rho(\lambda,\mu),p_{0},t)
	\left[
 \begin{array}
 {c}
 r+\alpha\tilde{x_{1}} \\
 v \\
 \end{array}
 \right]^{2} & = &
	[H_{xx}-H_{ux}H_{uu}^{-1}H_{xu}](\rho(\lambda,\mu),p_{0},t)(r+\alpha\tilde{x_{1}})^{2}
 \\
 & & +H_{uu}Z[\lambda,\mu]^{2}
 \end{eqnarray*}
 Because of assumption \ref{convexity} on the convexity, this
 proves that:
 \[
 	H_{\sigma \sigma}(\rho(\lambda,\mu),p_{0},t)
	\left[
 \begin{array}
 {c}
 r+\alpha\tilde{x_{1}} \\
 v \\
 \end{array}
 \right]^{2} \geq \beta Z[\lambda,\mu]^{2}
 \]
 so that:
 \begin{equation}
	\int_{0}^{T} \int_{0}^{1} \int_{0}^{1}
 H_{\sigma \sigma}(\rho(\lambda,\mu),p_{0},t)
	\left[
 \begin{array}
 {c}
 r+\alpha\tilde{x_{1}} \\
 v \\
 \end{array}
 \right]^{2} d \lambda d \mu dt
	\geq \beta z^{2}
 \label{J41bound}
 \end{equation}
 Including equations (\ref{J41bound}) and (\ref{J42bound}) into
 equation (\ref{expJ4}) leads to equation (\ref{J4LowerBound0}).
\end{proof}
\begin{proposition}\label{J41boundProp}
 Let's denote:
 \[
 J_{41} = 2\alpha \int_{0}^{T} \int_{0}^{1} \int_{0}^{1} \lambda \left[y_{1},v_{1}\right]
	[H_{\sigma \sigma}(\rho(\lambda,\mu),p_{0}t)-H_{\sigma \sigma}(w_{0},t)]
 \left[
 \begin{array}
 {c}
 r+\alpha\tilde{x_{1}} \\
 v \\
 \end{array}
 \right] d \lambda d \mu dt
 \]
 Then the following bound holds:
 \begin{equation}
 |J_{41}| \leq 4
 \sqrt{3}KM\alpha\left[\left((Tk_{r1}+\frac{1}{2}k_{v1}\right)z^{2}
 +\left(Tk_{r2}+\frac{1}{2}k_{v2}+\frac{T}{2}(M^{2}+2)\right)\alpha^{2}\right]
 \label{J41bound2}
 \end{equation}
\end{proposition}
\begin{proof}
 Let's expand $\left[y_{1},v_{1}\right]
	[H_{\sigma \sigma}(\rho(\lambda,\mu),p_{0}t)H_{\sigma \sigma}(w_{0},t)]
 \left[
 \begin{array}
 {c}
 r+\alpha\tilde{x_{1}} \\
 v \\
 \end{array}
 \right]$
	into its coordinates:
	\begin{equation}
		\begin{split}
	 & \left[y_{1},v_{1}\right]
	[H_{\sigma \sigma}(\rho(\lambda,\mu),p_{0}t)H_{\sigma \sigma}(w_{0},t)]
 \left[
 \begin{array}
 {c}
 r+\alpha\tilde{x_{1}} \\
 v
 \end{array}
 \right] \\ & =
	y_{1}[H_{xx}(\rho(\lambda,\mu),p_{0}t)-H_{xx}(w_{0},t)](r+\alpha\tilde{x_{1}})
 \\
 & + y_{1}[H_{xu}(\rho(\lambda,\mu),p_{0}t)-H_{xu}(w_{0},t)]v
 \label{J41exp}\\
 & + v_{1}[H_{ux}(\rho(\lambda,\mu),p_{0}t)-H_{ux}(w_{0},t)](r+\alpha\tilde{x_{1}})
 \\
 & +
	 v_{1}[H_{uu}(\rho(\lambda,\mu),p_{0}t)-H_{uu}(w_{0},t)]v
		 \end{split}
 \end{equation}
 But the second derivatives of $H(x,u,p_{0},t)$ are Lipschitz in
 $(x,u)$ of
 Lipschitz constant $K$ because the third derivatives of
 $H(x,u,p_{0},t)$ are bounded by $K$.
 Thus with the definition of $\rho(\lambda,\mu)$ and $w_{0}$,
 equation (\ref{J41exp}) leads to:
 \begin{align*}
 \left|\left[y_{1},v_{1}\right]
	 \right. & \left.
	[H_{\sigma \sigma}(\rho(\lambda,\mu),p_{0}t)H_{\sigma \sigma}(w_{0},t)]
 \left[
 \begin{array}
 {c}
 r+\alpha\tilde{x_{1}} \\
 v \\
 \end{array}
 \right] \right| \\
	&\leq
	\lambda\mu K \sqrt{(r+\alpha\tilde{x_{1}})^{2} + \alpha^{2}v_{1}^{2}}
	(|y_{1}||r+\alpha\tilde{x_{1}}| + |y_{1}||v| +
	|v||r+\alpha\tilde{x_{1}}|+|v_{1}||v|&
 \\
 & \leq K \sqrt{(r+\alpha)^{2} +
	 \alpha^{2}M^{2}}(2M(|r|+\alpha) + 2M|v|)
 \\
 & \leq 2MK \sqrt{2r^{2}+2\alpha^{2} +
	 M^{2}\alpha^{2}}\sqrt{3(r^{2}+v^{2}+\alpha^{2})}&
 \\
 & \leq 2MK \sqrt{3} \; (2r^{2}+v^{2}+(M^{2}+2)\alpha^{2})
 \end{align*}
	Thus, with a triple integration after multiplication by
	$\lambda$, we get:
	\[
	 |J_{41}| \leq 4
 \sqrt{3}KM\alpha\left[T\|r\|_{\infty}^{2} +
	 \frac{T}{2}\|v\|_{2}^{2}+((M^{2}+2)\alpha^{2}\right]
	\]
	Then, using equations (\ref{rinequality}) and
	(\ref{vinequality}) lead to equation (\ref{J41bound}).
\end{proof}
Applying propositions \ref{J4LowerBound0Prop} and \ref{J41boundProp}
lead to:
 \begin{eqnarray*}
 &&\int_{0}^{T} \int_{0}^{1} \int_{0}^{1} \lambda H_{\sigma
	\sigma}(\rho(\lambda,\mu),p_{0},t)(\delta \sigma)^{2}
	d \lambda d \mu dt \\ && \geq
	-2KM \left[TM+\sqrt{3}(2Tk_{r2}+k_{v2}+T(M^{2}+2))\alpha\right] \alpha^{2}
 \\
 & & + \left[\beta - 2\sqrt{3}KM(2Tk_{r1}+k_{v1})\alpha\right] z^{2}
 \\
	&&+ 2 \alpha \int_{0}^{T} \int_{0}^{1} \int_{0}^{1} \lambda \left[y_{1},v_{1}\right]
	H_{\sigma \sigma}(w_{0},t)
	\left[
	\begin{array}
	{c}
	 r+\alpha\tilde{x_{1}} \\
	 v \\
	\end{array}
	\right] d \lambda d \mu dt
 \end{eqnarray*}
That proves proposition \ref{J4LowerBoundProp} because
$H_{0 \sigma \sigma}=H_{\sigma \sigma}(w_{0},t)$
does not depend on $\lambda$ and $\mu$.

\section{Proof of Bound in absolute value for the sum of the integral
terms of the right hand sides of equations (\ref{J3LowerBound}) and
(\ref{J4LowerBound})}\label{JuLowerBoundProof}
Thanks to the fact that
$r+\alpha\tilde{x_{1}}= r_{1}+((r-r_{1})+\alpha\tilde{x_{1}})$,
 we have $R=R_{3}+R_{4}$, where $R_{3}$ and $R_{4}$ are defined as:
 \[
 R_{3}=\alpha \int_{0}^{T} \tilde{p_{1}}\left[f_{0x}r_{1} + f_{0u}v
	 + \left[y_{1},v_{1}\right]
	H_{0 \sigma \sigma}
	 \left[
	\begin{array}
	{c}
	 r_{1} \\
	 v \\
	\end{array}
	\right]\right] dt
 \]
and
 \[
 R_{4}=\alpha \int_{0}^{T} \left[y_{1},v_{1}\right]
	H_{0 \sigma \sigma}
	\left[
	\begin{array}
	{c}
	 r_{1}+\alpha\tilde{x_{1}} \\
	 0 \\
	\end{array}
	\right] dt
 \]

\begin{proposition}
 The following inequality holds:
 \begin{equation}
 |R_{3}| \leq kM\left((Tk_{r3}+k_{v1})\alpha z^{2}
	+(Tk_{r4}+k_{v2}) \alpha^{3}\right)
 \label{R3bound}
 \end{equation}
\end{proposition}
\begin{proof}
 Let's develop $\left[y_{1},v_{1}\right]
	H_{0 \sigma \sigma}
	 \left[
	\begin{array}
	{c}
	 r_{1} \\
	 v \\
	\end{array}
	\right]$ into coordinates:
	\[
	 \left[y_{1},v_{1}\right]
	H_{0 \sigma \sigma}
	 \left[
	\begin{array}
	{c}
	 r_{1} \\
	 v \\
	\end{array}
	\right]
	=H_{0xx}y_{1}r_{1} + H_{0ux}y_{1}v + H_{0xu}v_{1}r_{1} +
	H_{0uu}v_{1}v
	\]
	so that:
	\begin{align*}
	 \tilde{p_{1}}&\left[f_{0x}r_{1} + f_{0u}v
	 + \left[y_{1},v_{1}\right]
	H_{0 \sigma \sigma}
	 \left[
	\begin{array}
	{c}
	 r_{1} \\
	 v \\
	\end{array}
	\right]\right]\\
	& =(f_{0x}\tilde{p_{1}}+H_{0xx}y_{1} + H_{0xu}v_{1}) r_{1}
	+ (f_{0u}\tilde{p_{1}}+H_{0ux}y_{1} + H_{0uu}v_{1}) v
	\end{align*}

Now let's use the costate dynamics (\ref{auxcostate}) of the
auxiliary problem, together with its stationarity condition
(\ref{auxstationarity}). Then we get:
\[
 \tilde{p_{1}}\left[f_{0x}r_{1} + f_{0u}v
	 + \left[y_{1},v_{1}\right]
	H_{0 \sigma \sigma}
	 \left[
	\begin{array}
	{c}
	 r_{1} \\
	 v \\
	\end{array}
	\right]\right]
	= -\frac{dq_{1}}{dt}r_{1} - q_{1}\left[f_{0x}r_{1} + f_{0u}v\right]
\]
An integration by parts, together with the fact that
$r_{1}(0)=q_{1}(T)=0$ leads to:
\[
 R_{3}= \alpha \int_{0}^{T} q_{1}\left(\frac{dr_{1}}{dt}-\left[f_{0x}r_{1} + f_{0u}v\right]\right)
\]
But $r_{1}$ follows the dynamics (\ref{r1dynamics}), so that a Taylor
expansion of
$f(r_{1}+x_{0}+\alpha(\tilde{x_{1}}+y_{1}),v+u_{0}+\alpha v_{1},t)$
at $(x_{0}+\alpha(\tilde{x_{1}}+y_{1}),u_{0}+\alpha v_{1})$ leads to:
\[
 \left(\frac{dr_{1}}{dt}-\left[f_{0x}r_{1} + f_{0u}v\right]\right)
 = \int_{0}^{1} \int_{0}^{1} \lambda
 f_{\sigma\sigma}(x_{0}+\alpha(\tilde{x_{1}}+y_{1}) + \lambda\mu r_{1},
 u_{0}+\alpha v_{1} + \lambda\mu v,t)
 \left[
 \begin{array}
 {c}
 r_{1} \\
 v \\

 \end{array}
 \right]^{2} d\lambda d\mu
\]
so that, thanks to equation (\ref{fsigmasigma}) about the bound of
$f_{\sigma\sigma}$:
\[
 \left|q_{1}\left(\frac{dr_{1}}{dt}-\left[f_{0x}r_{1} + f_{0u}v\right]\right)\right|
 \leq \frac{1}{2}kM(|r_{1}|+|v|)^{2} \leq kM(r_{1}^{2}+v^{2})
\]
An integration on $[0,T]$ and a multiplication by $\alpha$ leads to
\[
 |R_{3}| \leq kM(T\|r_{1}\|_{\infty}^{2} + \|v\|_{2}^{2})
\]
That equation, together with the bounds on $r_{1}$ (\ref{r1inequality}) and $v$
(\ref{vinequality}) lead to equation (\ref{R3bound}).
\end{proof}

\begin{proposition}
 The following inequality holds:
 \begin{equation}
 |R_{4}| \leq 2KTM(1+k_{r5}\alpha) \alpha^{2}
 \label{R4bound}
 \end{equation}
\end{proposition}
\begin{proof}
 Let's develop $\left[y_{1},v_{1}\right]
	H_{0 \sigma \sigma}
	\left[
	\begin{array}
	{c}
	 r_{1}+\alpha\tilde{x_{1}} \\
	 0 \\
	\end{array}
	\right]$ into coordinates:
	\[
	 \left[y_{1},v_{1}\right]
	H_{0 \sigma \sigma}
	\left[
	\begin{array}
	{c}
	 r_{1}+\alpha\tilde{x_{1}} \\
	 0 \\
	\end{array}
	\right]
	=\left(H_{0xx}y_{1}+H_{0uu}v_{1}\right)((r-r_{1}) + \alpha \tilde{x_{1}})
	\]
	Thus, thanks to the bound on $r-r_{1}$ (\ref{rr1inequality}),
	its absolute value can be bounded:
	\[
	 \left|
	 \left[y_{1},v_{1}\right]
	H_{0 \sigma \sigma}
	\left[
	\begin{array}
	{c}
	 r_{1}+\alpha\tilde{x_{1}} \\
	 0 \\
	\end{array}
	\right]\right|
	\leq 2KM(\|r-r_{1}\|_{\infty}+\alpha)
	\leq 2KM(k_{r5}\alpha^{2}+\alpha)
	\leq 2KM(1+k_{r5}\alpha)\alpha
	\]
	An integration between $0$ and $T$ and a multiplication by
	$\alpha$ lead to equation (\ref{R4bound}).
\end{proof}

\section{Proof of equation (\ref{RestBound})}\label{RestBoundProof}
This is a consequence of equations (\ref{R3bound}) and
(\ref{R4bound}), together with the fact that $R=R_{3}+R_{4}$.

\bibliographystyle{plain}
\bibliography{Sandrine}

\end{document}